%%%%%%%%%%%%%%%%%%%%%%%%%%%%%%%%%%%%%%%%%%%%%%%%%%%%%%%%%%%%%%%%%%%%%%%%%%%%
%
%A  colour.tex              GRAPE documentation              Leonard Soicher
%
%
%
\def\GRAPE{\sf GRAPE}
\def\nauty{\it nauty}
\def\G{\Gamma}
\def\Aut{{\rm Aut}\,}
\def\x{\times}
\Chapter{Vertex-Colouring and Complete Subgraphs}

The following sections describe functions for (proper) vertex-colouring
and determining complete subgraphs of a given simple graph. Included are
functions for determining the chromatic number and the clique number of
a simple graph. The methods used for proper vertex-colouring are described
in \cite{Soi24a}.

The function `CompleteSubgraphsOfGivenSize' can be used to determine
the complete subgraphs with given vertex-weight sum in a vertex-weighted
graph,
\index{vertex-weighted graph}
where the weights can be positive
integers or non-zero vectors of non-negative integers.

%%%%%%%%%%%%%%%%%%%%%%%%%%%%%%%%%%%%%%%%%%%%%%%%%%%%%%%%%%%%%%%%%%%%%%%%
\Section{VertexColouring}

\>VertexColouring( <gamma> )
\>VertexColouring( <gamma>, <k> )
\>VertexColouring( <gamma>, <k>, <m> )

This function returns a proper vertex-colouring <C> for the graph
<gamma>, which must be simple. A *proper vertex-colouring*
\index{proper vertex-colouring} 
of <gamma> is an assignment of colours to the vertices
of <gamma>, such that, if $[i,j]$ is an edge, then vertices $i$ and $j$
are assigned different colours.

The returned proper vertex-colouring <C> is given as a list of positive
integers (the *colours*), indexed by the vertices of <gamma>, with the
property that $<C>[i]\not=<C>[j]$ whenever $[i,j]$ is an edge of <gamma>.

If the optional parameter <k> is given, then <k> must be a non-negative
integer. In this case, a proper vertex-colouring using at most <k>
colours is returned, if such a colouring exists, and `fail' otherwise.

If, in addition to <k>, the optional parameter <m> is given, then <m>
must be a a non-negative integer, such that there is no monochromatic
set of vertices of size greater than <m> in any proper vertex-colouring
of <gamma> which uses at most <k> colours.  This information (which is
not checked) may help to speed up the function.

\beginexample
gap> J:=JohnsonGraph(5,2);
rec( adjacencies := [ [ 2, 3, 4, 5, 6, 7 ] ], group := Group([ (1,5,8,10,4)
  (2,6,9,3,7), (2,5)(3,6)(4,7) ]), isGraph := true, isSimple := true, 
  names := [ [ 1, 2 ], [ 1, 3 ], [ 1, 4 ], [ 1, 5 ], [ 2, 3 ], [ 2, 4 ], 
      [ 2, 5 ], [ 3, 4 ], [ 3, 5 ], [ 4, 5 ] ], order := 10, 
  representatives := [ 1 ], schreierVector := [ -1, 2, 2, 1, 1, 1, 2, 1, 1, 1 
     ] )
gap> VertexColouring(J);
[ 1, 3, 5, 4, 2, 3, 6, 1, 5, 2 ]
gap> VertexColouring(J,5);
[ 1, 2, 3, 4, 5, 4, 2, 1, 3, 5 ]
gap> VertexColouring(J,4);
fail
\endexample

%%%%%%%%%%%%%%%%%%%%%%%%%%%%%%%%%%%%%%%%%%%%%%%%%%%%%%%%%%%%%%%%%%%%%%%%
\Section{IsVertexColouring}

\>IsVertexColouring( <gamma>, <C> )
\>IsVertexColouring( <gamma>, <C>, <k> )

Suppose that <gamma> is a simple graph, <C> is a list of positive integers
of length `OrderGraph(<gamma>)', and <k> is a non-negative integer
(default: `OrderGraph(<gamma>)').

Then this function returns `true' if <C> is a vertex <k>-colouring of
<gamma>, that is, a proper vertex-colouring using at most <k>-colours (for
which $<C>[i]$ is the colour of the $i$-th vertex), and `false' if not.

See also "VertexColouring".

\beginexample
gap> gamma:=JohnsonGraph(7,3);;
gap> C:=VertexColouring(gamma,6);;
gap> IsVertexColouring(gamma,C);
true
gap> IsVertexColouring(gamma,C,7);
true
gap> IsVertexColouring(gamma,C,6);
true
gap> IsVertexColouring(gamma,C,5);
false
\endexample

%%%%%%%%%%%%%%%%%%%%%%%%%%%%%%%%%%%%%%%%%%%%%%%%%%%%%%%%%%%%%%%%%%%%%%%%
\Section{MinimumVertexColouring}

\>MinimumVertexColouring( <gamma> )

This function returns a minimum vertex-colouring <C> for the graph
<gamma>, which must be simple. A *minimum vertex-colouring*
\index{minimum vertex-colouring} 
is a proper vertex-colouring using as few colours as possible.

The returned minimum vertex-colouring <C> is given as a list of positive
integers (the *colours*), indexed by the vertices of <gamma>, with the
property that $<C>[i]\not=<C>[j]$ whenever $[i,j]$ is an edge of <gamma>,
and subject to this property, the number of distinct elements of <C>
is as small as possible.

See also "VertexColouring".

\beginexample
gap> J:=JohnsonGraph(5,2);
rec( adjacencies := [ [ 2, 3, 4, 5, 6, 7 ] ], group := Group([ (1,5,8,10,4)
  (2,6,9,3,7), (2,5)(3,6)(4,7) ]), isGraph := true, isSimple := true, 
  names := [ [ 1, 2 ], [ 1, 3 ], [ 1, 4 ], [ 1, 5 ], [ 2, 3 ], [ 2, 4 ], 
      [ 2, 5 ], [ 3, 4 ], [ 3, 5 ], [ 4, 5 ] ], order := 10, 
  representatives := [ 1 ], schreierVector := [ -1, 2, 2, 1, 1, 1, 2, 1, 1, 1 
     ] )
gap> MinimumVertexColouring(J);
[ 1, 2, 3, 4, 5, 4, 2, 1, 3, 5 ]
\endexample

%%%%%%%%%%%%%%%%%%%%%%%%%%%%%%%%%%%%%%%%%%%%%%%%%%%%%%%%%%%%%%%%%%%%%%%%
\Section{ChromaticNumber}

\>ChromaticNumber( <gamma> )

This function returns the chromatic number of the given graph <gamma>,
which must be simple.  The *chromatic number*
\index{chromatic number} 
of <gamma> is the minimum number of colours needed to properly vertex-colour
<gamma>, that is, the number of colours used in a minimum vertex-colouring
of <gamma>.

See also "MinimumVertexColouring".

\beginexample
gap> ChromaticNumber(JohnsonGraph(5,2));
5
gap> ChromaticNumber(JohnsonGraph(6,2));
5
gap> ChromaticNumber(JohnsonGraph(7,2));
7
\endexample

%%%%%%%%%%%%%%%%%%%%%%%%%%%%%%%%%%%%%%%%%%%%%%%%%%%%%%%%%%%%%%%%%%%%%%%%
\Section{CompleteSubgraphs}

\>CompleteSubgraphs( <gamma> )
\>CompleteSubgraphs( <gamma>, <k> )
\>CompleteSubgraphs( <gamma>, <k>, <alls> )

Let <gamma> be a simple graph and <k> an integer. This function returns
a set <K> of complete subgraphs of <gamma>, where a complete subgraph is
represented by its vertex-set.  If <k> is non-negative then the elements
of <K> each have size <k>, otherwise the elements of <K> represent maximal
complete subgraphs of <gamma>. (A *maximal complete subgraph* of <gamma>
is a complete subgraph of <gamma> which is not properly contained in
another complete subgraph of <gamma>.) The default for <k> is $-1$,
i.e. maximal complete subgraphs.  See also `CompleteSubgraphsOfGivenSize',
which can be used to compute the maximal complete subgraphs of given
size, and can also be used to determine the (maximal or otherwise)
complete subgraphs with given vertex-weight sum in a vertex-weighted
graph.

The optional parameter <alls> controls how many complete subgraphs are
returned. The valid values for <alls> are 0, 1 (the default), and 2.

*Warning:* Using the default value of 1 for <alls> (see below) means that
more than one element may be returned for some `<gamma>.group' orbit(s)
of the required complete subgraphs.  To obtain just one element from each
`<gamma>.group' orbit of the required complete subgraphs, you must give
the value 2 to the parameter <alls>.

If <alls>=0 (or `false' for backward compatibility) then <K> will contain
at most one element. In this case, if <k> is negative then <K> will
contain just one maximal complete subgraph, and if <k> is non-negative
then <K> will contain a complete subgraph of size <k> if and only if
such a subgraph is contained in <gamma>.

If <alls>=1 (or `true' for backward compatibility) then <K> will contain
(perhaps properly) a set of `<gamma>.group' orbit-representatives of
the maximal (if <k> is negative) or size <k> (if <k> is non-negative)
complete subgraphs of <gamma>.

If <alls>=2 then <K> will be a set of `<gamma>.group'
orbit-representatives of the maximal (if <k> is negative) or size <k>
(if <k> is non-negative) complete subgraphs of <gamma>.  This option
can be more costly than when <alls>=1.

Before applying `CompleteSubgraphs', one may want to associate the full
automorphism group of <gamma> with <gamma>, via `<gamma> :=
NewGroupGraph( AutGroupGraph(<gamma>), <gamma> );'.

An alternative name for this function is `Cliques'.
\index{Cliques}

See also "CompleteSubgraphsOfGivenSize".

\beginexample
gap> gamma := JohnsonGraph(5,2);
rec( isGraph := true, order := 10, 
  group := Group([ ( 1, 5, 8,10, 4)( 2, 6, 9, 3, 7), ( 2, 5)( 3, 6)( 4, 7) ]),
  schreierVector := [ -1, 2, 2, 1, 1, 1, 2, 1, 1, 1 ], 
  adjacencies := [ [ 2, 3, 4, 5, 6, 7 ] ], representatives := [ 1 ], 
  names := [ [ 1, 2 ], [ 1, 3 ], [ 1, 4 ], [ 1, 5 ], [ 2, 3 ], [ 2, 4 ], 
      [ 2, 5 ], [ 3, 4 ], [ 3, 5 ], [ 4, 5 ] ], isSimple := true )
gap> CompleteSubgraphs(gamma);
[ [ 1, 2, 3, 4 ], [ 1, 2, 5 ] ]
gap>  CompleteSubgraphs(gamma,3,2);
[ [ 1, 2, 3 ], [ 1, 2, 5 ] ]
gap> CompleteSubgraphs(gamma,-1,0);
[ [ 1, 2, 5 ] ]
\endexample

%%%%%%%%%%%%%%%%%%%%%%%%%%%%%%%%%%%%%%%%%%%%%%%%%%%%%%%%%%%%%%%%%%%%%%%%
\Section{CompleteSubgraphsOfGivenSize}

\>CompleteSubgraphsOfGivenSize( <gamma>, <k> )
\>CompleteSubgraphsOfGivenSize( <gamma>, <k>, <alls> )
\>CompleteSubgraphsOfGivenSize( <gamma>, <k>, <alls>, <maxi> )
\>CompleteSubgraphsOfGivenSize( <gamma>, <k>, <alls>, <maxi>, <col> )
\>CompleteSubgraphsOfGivenSize( <gamma>, <k>, <alls>, <maxi>, <col>, <wts> ) 

Let <gamma> be a simple graph, and <k> a non-negative integer or vector
of non-negative integers.  This function returns a set <K> (possibly
empty) of complete subgraphs of size <k> of <gamma>.  The vertices may
have weights, which should be non-zero integers if <k> is an integer and
non-zero $d$-vectors of non-negative integers if <k> is a $d$-vector,
and in these cases, a complete subgraph of *size* <k> means a complete
subgraph whose vertex-weights sum to <k>.  The exact nature of the set
<K> depends on the values of the parameters supplied to this function. A
complete subgraph is represented by its vertex-set.

The optional parameter <alls> controls how many complete subgraphs are
returned. The valid values for <alls> are 0, 1 (the default), and 2.

*Warning:* Using the default value of 1 for <alls> (see below) means that
more than one element may be returned for some `<gamma>.group' orbit(s)
of the required complete subgraphs.  To obtain just one element from each
`<gamma>.group' orbit of the required complete subgraphs, you must give
the value 2 to the parameter <alls>.

If <alls>=0 (or `false' for backward compatibility) then <K> will
contain at most one element.  If <maxi>=`false' then <K> will contain one
element if and only if <gamma> contains a complete subgraph of size <k>.
If <maxi>=`true' then <K> will contain one element if and only if <gamma>
contains a maximal complete subgraph of size <k>, in which case <K>
will contain (the vertex-set of) such a maximal complete subgraph.
(A *maximal complete subgraph* of <gamma> is a complete subgraph of
<gamma> which is not properly contained in another complete subgraph
of <gamma>.)

If <alls>=1 (or `true' for backward compatibility) and <maxi>=`false',
then <K> will contain (perhaps properly) a set of `<gamma>.group'
orbit-representatives of the size <k> complete subgraphs of <gamma>.
If <alls>=1 (the default) and <maxi>=`true', then <K> will contain
(perhaps properly) a set of `<gamma>.group' orbit-representatives of
the size <k> *maximal* complete subgraphs of <gamma>.

If <alls>=2 and <maxi>=`false', then <K> will be a set of `<gamma>.group'
orbit-representatives of the size <k> complete subgraphs of <gamma>.
If <alls>=2 and <maxi>=`true' then <K> will be a set of `<gamma>.group'
orbit-representatives of the size <k> *maximal* complete subgraphs
of <gamma>.  This option can be more costly than when <alls>=1.

The optional parameter <maxi> controls whether only maximal complete
subgraphs of size <k> are returned.  The default is `false', which means
that non-maximal as well as maximal complete subgraphs of size <k> are
returned. If <maxi>=`true' then only maximal complete subgraphs of size
<k> are returned. (Previous to version 4.1 of {\GRAPE}, <maxi>=`true'
meant that it was assumed (but not checked) that all complete subgraphs
of size <k> were maximal.)

The optional boolean parameter <col> is used to determine whether or not
partial proper vertex-colouring is used to cut down the search tree. The
default is `true', which says to use this partial colouring.  For backward
compatibility, <col> a rational number means the same as <col>=`true'.

The optional parameter <wts> should be a list of vertex-weights; the list
should be of length `<gamma>.order', with the <i>-th element being the
weight of vertex <i>. The weights must be all positive integers if <k>
is an integer, and all non-zero $d$-vectors of non-negative integers
if <k> is a $d$-vector. The default is that all weights are equal to~1.
(Recall that a complete subgraph of *size* <k> means a complete subgraph
whose vertex-weights sum to <k>.)

If <wts> is a list of (positive) integers, then it is required that
for all $g$ in `<gamma>.group' and all $v$ in `Vertices(<gamma>)',
we have $<wts>[v^g]=<wts>[v]$.

If <wts> is a list of $d$-vectors then we assume that there is some group
$G$ and epimorphism $\theta$ from $G$ to `<gamma>.group', such that there
is an action $\mu$ of $G$ on `[1..$d$]', giving an action of $G$ on the
set of integer $d$-vectors, where if $w$ is an integer $d$-vector and
$g$ in $G$ then $w^g$ is defined by $w^g[\mu(i,g)]=w[i]$ for all $i$
in `[1..$d$]'. It is then required that for all $g$ in $G$, we have
$<k>^g=<k>$ and for all $v$ in `Vertices(<gamma>)', $<wts>[v^{g\theta}]
= <wts>[v]^g$.  These requirements are *not* checked by the function,
and the use of vector-weights is primarily for the {\DESIGN} package
and advanced users of {\GRAPE}.

An alternative name for this function is 
`CliquesOfGivenSize'.
\index{CliquesOfGivenSize}

See also "CompleteSubgraphs".

\beginexample
gap> gamma:=JohnsonGraph(6,2);                         
rec( isGraph := true, order := 15, 
  group := Group([ ( 1, 6,10,13,15, 5)( 2, 7,11,14, 4, 9)( 3, 8,12), 
      ( 2, 6)( 3, 7)( 4, 8)( 5, 9) ]), 
  schreierVector := [ -1, 2, 2, 1, 1, 1, 1, 1, 2, 1, 1, 1, 1, 1, 1 ], 
  adjacencies := [ [ 2, 3, 4, 5, 6, 7, 8, 9 ] ], representatives := [ 1 ], 
  names := [ [ 1, 2 ], [ 1, 3 ], [ 1, 4 ], [ 1, 5 ], [ 1, 6 ], [ 2, 3 ], 
      [ 2, 4 ], [ 2, 5 ], [ 2, 6 ], [ 3, 4 ], [ 3, 5 ], [ 3, 6 ], [ 4, 5 ], 
      [ 4, 6 ], [ 5, 6 ] ], isSimple := true )
gap> CompleteSubgraphsOfGivenSize(gamma,4);
[ [ 1, 2, 3, 4 ] ]
gap> CompleteSubgraphsOfGivenSize(gamma,4,1,true);
[  ]
gap> CompleteSubgraphsOfGivenSize(gamma,5,2,true);
[ [ 1, 2, 3, 4, 5 ] ]
gap> delta:=NewGroupGraph(Group(()),gamma);;
gap> CompleteSubgraphsOfGivenSize(delta,5,2,true);
[ [ 1, 2, 3, 4, 5 ], [ 1, 6, 7, 8, 9 ], [ 2, 6, 10, 11, 12 ], 
  [ 3, 7, 10, 13, 14 ], [ 4, 8, 11, 13, 15 ], [ 5, 9, 12, 14, 15 ] ]
gap> CompleteSubgraphsOfGivenSize(delta,5,0);
[ [ 1, 2, 3, 4, 5 ] ]
gap> CompleteSubgraphsOfGivenSize(delta,5,1,false,true,
>       [1,2,3,4,5,6,7,8,7,6,5,4,3,2,1]);
[ [ 1, 4 ], [ 2, 3 ], [ 3, 14 ], [ 4, 15 ], [ 5 ], [ 11 ], [ 12, 15 ], 
  [ 13, 14 ] ]
\endexample
%%%%%%%%%%%%%%%%%%%%%%%%%%%%%%%%%%%%%%%%%%%%%%%%%%%%%%%%%%%%%%%%%%%%%%%%

\Section{MaximumClique}

\>MaximumClique( <gamma> )

This function returns a maximum clique of the graph <gamma>, which must
be simple.  A *maximum clique*
\index{maximum clique} 
of <gamma> is a
set of pairwise adjacent vertices of <gamma> of the largest possible size.

An alternative name for this function is
`MaximumCompleteSubgraph'.
\index{MaximumCompleteSubgraph} 

See also "CompleteSubgraphsOfGivenSize".

\beginexample
gap> J:=JohnsonGraph(5,2);
rec( adjacencies := [ [ 2, 3, 4, 5, 6, 7 ] ], group := Group([ (1,5,8,10,4)
  (2,6,9,3,7), (2,5)(3,6)(4,7) ]), isGraph := true, isSimple := true, 
  names := [ [ 1, 2 ], [ 1, 3 ], [ 1, 4 ], [ 1, 5 ], [ 2, 3 ], [ 2, 4 ], 
      [ 2, 5 ], [ 3, 4 ], [ 3, 5 ], [ 4, 5 ] ], order := 10, 
  representatives := [ 1 ], schreierVector := [ -1, 2, 2, 1, 1, 1, 2, 1, 1, 1 
     ] )
gap> MaximumClique(J);
[ 1, 2, 3, 4 ]
\endexample

%%%%%%%%%%%%%%%%%%%%%%%%%%%%%%%%%%%%%%%%%%%%%%%%%%%%%%%%%%%%%%%%%%%%%%%%
\Section{CliqueNumber}

\>CliqueNumber( <gamma> )

This function returns the clique number of the given graph <gamma>,
which must be simple.  The *clique number*
\index{clique number} 
of <gamma> is the size of a largest clique in <gamma>, where a *clique*
is a set of pairwise adjacent vertices.

\beginexample
gap> CliqueNumber(JohnsonGraph(5,2));
4
gap> CliqueNumber(JohnsonGraph(6,2));
5
gap> CliqueNumber(JohnsonGraph(7,2));
6
\endexample
