%%%%%%%%%%%%%%%%%%%%%%%%%%%%%%%%%%%%%%%%%%%%%%%%%%%%%%%%%%%%%%%%%%%%%%%%%%%%
%
%A  cnauty.tex              GRAPE documentation              Leonard Soicher
%
%
%
\def\GRAPE{\sf GRAPE}
\def\nauty{\it nauty}
\def\bliss{\it bliss}
\def\G{\Gamma}
\def\Aut{{\rm Aut}\,}
\def\x{\times}
\Chapter{Automorphism groups and isomorphism testing for graphs}

{\GRAPE} includes B.~D.~McKay{\pif}s {\nauty} (Version~2.8.6) package
for calculating automorphism groups of graphs and for testing graph
isomorphism (see \cite{MP14}).  As described in Section "Installing the
GRAPE Package", a user may instead use their own copy of nauty/dreadnaut,
or may use T.~Juntilla{\pif}s and P.~Kaski{\pif}s {\bliss} package
\cite{JK07} instead of {\nauty}. Many functions described in this chapter
make use of {\nauty} or {\bliss}.

%%%%%%%%%%%%%%%%%%%%%%%%%%%%%%%%%%%%%%%%%%%%%%%%%%%%%%%%%%%%%%%%%%%%%%%%
\Section{Graphs with colour-classes}

For each of the functions described in this chapter, each graph parameter
may be replaced by a graph with colour-classes, which is a record having
(at least) the components `graph' (which should be a graph in {\GRAPE}
format), and `colourClasses', which should be an ordered partition of the
vertices of the graph, and so define *colour-classes* for the vertices.
This ordered partition should be given as a list of (pairwise-disjoint
non-empty) sets partitioning the vertex-set.  When these functions are
called with graphs with colour-classes, then it is understood that an
*automorphism* of a graph with colour-classes is an automorphism of the
graph which additionally preserves the list of colour-classes (classwise),
and an *isomorphism* from one graph with colour-classes to a second is a
graph isomorphism from the first graph to the second which additionally
maps the first list of colour-classes to the second (classwise). The
record for a graph with colour-classes may also optionally contain the
additional components `autGroup' and/or `canonicalLabelling', and these
are handled in an analogous way to those for a graph (such as when using
the parameter <firstunbindcanon>).  Note that we do not require that
adjacent vertices be in different colour-classes.

%%%%%%%%%%%%%%%%%%%%%%%%%%%%%%%%%%%%%%%%%%%%%%%%%%%%%%%%%%%%%%%%%%%%%%%%
\Section{AutGroupGraph}

\>AutGroupGraph( <gamma> )
\>AutGroupGraph( <gamma>, <colourclasses> )

The first version of this function returns the automorphism group of
the graph (or graph with colour-classes) <gamma>, using {\nauty} 
or {\bliss} (this can also be accomplished by typing
`AutomorphismGroup(<gamma>)'). The *automorphism group* $\Aut(<gamma>)$
of a graph <gamma> is the group consisting of the permutations
of the vertices of <gamma> which preserve the edge-set of <gamma>.
The *automorphism group* of a graph with colour-classes is the subgroup
of the automorphism group of the graph which preserves the colour-classes
(classwise).

The second version of this function is maintained only for backward
compatibility. For this version <gamma> must be a graph, <colourclasses>
is an ordered partition of the vertices of <gamma>, and the subgroup of
$\Aut(<gamma>)$ preserving this ordered partition is returned. The ordered
partition should be given as a list of (pairwise-disjoint non-empty) sets
partitioning the vertices of <gamma>, although for backward compatibility
and only in this situation, the last set in the ordered partition need
not be included explicitly.

\beginexample
gap> gamma := JohnsonGraph(4,2);                   
rec( adjacencies := [ [ 2, 3, 4, 5 ] ], 
  group := Group([ (1,4,6,3)(2,5), (2,4)(3,5) ]), isGraph := true, 
  isSimple := true, 
  names := [ [ 1, 2 ], [ 1, 3 ], [ 1, 4 ], [ 2, 3 ], [ 2, 4 ], [ 3, 4 ] ], 
  order := 6, representatives := [ 1 ], 
  schreierVector := [ -1, 2, 1, 1, 1, 1 ] )
gap> Size(AutGroupGraph(gamma)); 
48
gap> AutGroupGraph( rec(graph:=gamma,colourClasses:=[[1,2,3],[4,5,6]]) ); 
Group([ (2,3)(4,5), (1,2)(5,6) ])
gap> Size(AutomorphismGroup( rec(graph:=gamma,colourClasses:=[[1,6],[2,3,4,5]]) )); 
16
\endexample

%%%%%%%%%%%%%%%%%%%%%%%%%%%%%%%%%%%%%%%%%%%%%%%%%%%%%%%%%%%%%%%%%%%%%%%%
\Section{GraphIsomorphism}

\>GraphIsomorphism( <gamma1>, <gamma2> )
\>GraphIsomorphism( <gamma1>, <gamma2>, <firstunbindcanon> )

Let <gamma1> and <gamma2> both be graphs or both be graphs with
colour-classes.  Then this function makes use of {\nauty} or {\bliss} to
(try to) determine an isomorphism from <gamma1> to <gamma2>.  If <gamma1>
and <gamma2> are isomorphic, then this function returns an isomorphism
from <gamma1> to <gamma2>. This isomorphism will be a permutation of the
vertices of <gamma1> which maps the edge-set of <gamma1> onto that of
<gamma2>, and if <gamma1> and <gamma2> are graphs with colour-classes,
this isomorphism will also map the colour-class list of <gamma1> to that
of <gamma2> (classwise). If <gamma1> and <gamma2> are not isomorphic
then this function returns `fail'.

The optional boolean parameter <firstunbindcanon> determines whether or
not the `canonicalLabelling' components of both <gamma1> and <gamma2>
are first unbound before proceeding.  If <firstunbindcanon> is `true'
(the default, safe and possibly much slower option) then these components
are first unbound.  If <firstunbindcanon> is `false', then any existing
`canonicalLabelling' components are used.  However, since canonical
labellings can depend on whether {\nauty} or {\bliss} is used, the version
of {\nauty} or {\bliss} used, the version of {\GRAPE}, parameter settings
of {\nauty} or {\bliss}, and possibly even the compiler and computer
used, you must be sure that if <firstunbindcanon>=`false' then the
`canonicalLabelling' component(s) which may already exist for <gamma1>
or <gamma2> were created in exactly the same environment in which you
are presently computing.

Please also note that a canonical labelling for a {\GRAPE} graph is the
inverse of what a canononical labelling for a graph is usually defined as
(such as in {\bliss}), in that in {\GRAPE}, the image of a graph under
the *inverse* of its canonical labelling is the calculated canonical
version of that graph.

See also "IsIsomorphicGraph".

\beginexample
gap> gamma := JohnsonGraph(5,3);
rec( adjacencies := [ [ 2, 3, 4, 5, 7, 8 ] ], 
  group := Group([ (1,7,10,6,3)(2,8,4,9,5), (4,7)(5,8)(6,9) ]), 
  isGraph := true, isSimple := true, 
  names := [ [ 1, 2, 3 ], [ 1, 2, 4 ], [ 1, 2, 5 ], [ 1, 3, 4 ], [ 1, 3, 5 ], 
      [ 1, 4, 5 ], [ 2, 3, 4 ], [ 2, 3, 5 ], [ 2, 4, 5 ], [ 3, 4, 5 ] ], 
  order := 10, representatives := [ 1 ], 
  schreierVector := [ -1, 1, 1, 2, 1, 1, 1, 2, 1, 1 ] )
gap> delta := JohnsonGraph(5,2);
rec( adjacencies := [ [ 2, 3, 4, 5, 6, 7 ] ], 
  group := Group([ (1,5,8,10,4)(2,6,9,3,7), (2,5)(3,6)(4,7) ]), 
  isGraph := true, isSimple := true, 
  names := [ [ 1, 2 ], [ 1, 3 ], [ 1, 4 ], [ 1, 5 ], [ 2, 3 ], [ 2, 4 ], 
      [ 2, 5 ], [ 3, 4 ], [ 3, 5 ], [ 4, 5 ] ], order := 10, 
  representatives := [ 1 ], schreierVector := [ -1, 2, 2, 1, 1, 1, 2, 1, 1, 1 
     ] )
gap> GraphIsomorphism( gamma, delta );
(3,5,6,8,7,4)
gap> GraphIsomorphism( 
>       rec(graph:=gamma, colourClasses:=[[7],[1,2,3,4,5,6,8,9,10]]), 
>       rec(graph:=delta, colourClasses:=[[10],[1..9]]) ); 
(1,3)(2,6,5)(4,8)(7,10,9)
gap> GraphIsomorphism( 
>       rec(graph:=gamma, colourClasses:=[[1],[6],[2,3,4,5,7,8,9,10]]), 
>       rec(graph:=delta, colourClasses:=[[1],[6],[2,3,4,5,7,8,9,10]]) ); 
fail
\endexample

%%%%%%%%%%%%%%%%%%%%%%%%%%%%%%%%%%%%%%%%%%%%%%%%%%%%%%%%%%%%%%%%%%%%%%%%
\Section{IsIsomorphicGraph}

\>IsIsomorphicGraph( <gamma1>, <gamma2> )
\>IsIsomorphicGraph( <gamma1>, <gamma2>, <firstunbindcanon> )

Let <gamma1> and <gamma2> both be graphs or both be graphs with
colour-classes.  Then this boolean function makes use of the {\nauty} or 
{\bliss} package to test whether <gamma1> and <gamma2> are isomorphic 
(as graphs or as graphs with colour-classes, respectively). The
value `true' is returned if and only if the graphs (or graphs with
colour-classes) are isomorphic.

The optional boolean parameter <firstunbindcanon> determines whether or
not the `canonicalLabelling' components of both <gamma1> and <gamma2>
are first unbound before proceeding.  If <firstunbindcanon> is `true'
(the default, safe and possibly much slower option) then these components
are first unbound.  If <firstunbindcanon> is `false', then any existing
`canonicalLabelling' components are used.  However, since canonical
labellings can depend on whether {\nauty} or {\bliss} is used, the version
of {\nauty} or {\bliss} used, the version of {\GRAPE}, parameter settings
of {\nauty} or {\bliss}, and possibly even the compiler and computer
used, you must be sure that if <firstunbindcanon>=`false' then the
`canonicalLabelling' component(s) which may already exist for <gamma1>
or <gamma2> were created in exactly the same environment in which you
are presently computing.

See also "GraphIsomorphism".  For pairwise isomorphism testing
of three or more graphs (or graphs with colour-classes), see
"GraphIsomorphismClassRepresentatives".

Please also note that a canonical labelling for a {\GRAPE} graph is the
inverse of what a canononical labelling for a graph is usually defined as
(such as in {\bliss}), in that in {\GRAPE}, the image of a graph under
the *inverse* of its canonical labelling is the calculated canonical
version of that graph.

\beginexample
gap> gamma := JohnsonGraph(5,3);
rec( adjacencies := [ [ 2, 3, 4, 5, 7, 8 ] ], 
  group := Group([ (1,7,10,6,3)(2,8,4,9,5), (4,7)(5,8)(6,9) ]), 
  isGraph := true, isSimple := true, 
  names := [ [ 1, 2, 3 ], [ 1, 2, 4 ], [ 1, 2, 5 ], [ 1, 3, 4 ], [ 1, 3, 5 ], 
      [ 1, 4, 5 ], [ 2, 3, 4 ], [ 2, 3, 5 ], [ 2, 4, 5 ], [ 3, 4, 5 ] ], 
  order := 10, representatives := [ 1 ], 
  schreierVector := [ -1, 1, 1, 2, 1, 1, 1, 2, 1, 1 ] )
gap> delta := JohnsonGraph(5,2);
rec( adjacencies := [ [ 2, 3, 4, 5, 6, 7 ] ], 
  group := Group([ (1,5,8,10,4)(2,6,9,3,7), (2,5)(3,6)(4,7) ]), 
  isGraph := true, isSimple := true, 
  names := [ [ 1, 2 ], [ 1, 3 ], [ 1, 4 ], [ 1, 5 ], [ 2, 3 ], [ 2, 4 ], 
      [ 2, 5 ], [ 3, 4 ], [ 3, 5 ], [ 4, 5 ] ], order := 10, 
  representatives := [ 1 ], schreierVector := [ -1, 2, 2, 1, 1, 1, 2, 1, 1, 1 
     ] )
gap> IsIsomorphicGraph( gamma, delta );
true
gap> IsIsomorphicGraph( 
>       rec(graph:=gamma, colourClasses:=[[7],[1,2,3,4,5,6,8,9,10]]), 
>       rec(graph:=delta, colourClasses:=[[10],[1..9]]) ); 
true
gap> IsIsomorphicGraph( 
>       rec(graph:=gamma, colourClasses:=[[1],[6],[2,3,4,5,7,8,9,10]]), 
>       rec(graph:=delta, colourClasses:=[[1],[6],[2,3,4,5,7,8,9,10]]) ); 
false
\endexample

%%%%%%%%%%%%%%%%%%%%%%%%%%%%%%%%%%%%%%%%%%%%%%%%%%%%%%%%%%%%%%%%%%%%%%%%
\Section{GraphIsomorphismClassRepresentatives}

\>GraphIsomorphismClassRepresentatives( <L> )
\>GraphIsomorphismClassRepresentatives( <L>, <firstunbindcanon> )

Given a list <L> of graphs, or of graphs with colour-classes, this
function uses {\nauty} or {\bliss} to return a list consisting of pairwise
non-isomorphic elements of <L>, representing all the isomorphism classes
of elements of <L>.

The optional boolean parameter <firstunbindcanon> determines whether
or not the `canonicalLabelling' components of all elements of <L>
are first unbound before proceeding.  If <firstunbindcanon> is `true'
(the default, safe and possibly slower option) then these components
are first unbound.  If <firstunbindcanon> is `false', then any existing
`canonicalLabelling' components of elements of <L> are used.  However,
since canonical labellings can depend on whether {\nauty} or {\bliss} is
used, the version of {\nauty} or {\bliss} used, the version of {\GRAPE},
parameter settings of {\nauty} or {\bliss}, and possibly even the compiler
and computer used, you must be sure that if <firstunbindcanon>=`false'
then any `canonicalLabelling' component(s) which may already exist for
elements of <L> were created in exactly the same environment in which
you are presently computing.

It is assumed that the computing environment is constant throughout the 
execution of this function.

Please also note that a canonical labelling for a {\GRAPE} graph is the
inverse of what a canononical labelling for a graph is usually defined as
(such as in {\bliss}), in that in {\GRAPE}, the image of a graph under
the *inverse* of its canonical labelling is the calculated canonical
version of that graph.

\beginexample 
gap> A:=JohnsonGraph(5,3);
rec( adjacencies := [ [ 2, 3, 4, 5, 7, 8 ] ], 
  group := Group([ (1,7,10,6,3)(2,8,4,9,5), (4,7)(5,8)(6,9) ]), 
  isGraph := true, isSimple := true, 
  names := [ [ 1, 2, 3 ], [ 1, 2, 4 ], [ 1, 2, 5 ], [ 1, 3, 4 ], [ 1, 3, 5 ], 
      [ 1, 4, 5 ], [ 2, 3, 4 ], [ 2, 3, 5 ], [ 2, 4, 5 ], [ 3, 4, 5 ] ], 
  order := 10, representatives := [ 1 ], 
  schreierVector := [ -1, 1, 1, 2, 1, 1, 1, 2, 1, 1 ] )
gap> B:=JohnsonGraph(5,2);
rec( adjacencies := [ [ 2, 3, 4, 5, 6, 7 ] ], 
  group := Group([ (1,5,8,10,4)(2,6,9,3,7), (2,5)(3,6)(4,7) ]), 
  isGraph := true, isSimple := true, 
  names := [ [ 1, 2 ], [ 1, 3 ], [ 1, 4 ], [ 1, 5 ], [ 2, 3 ], [ 2, 4 ], 
      [ 2, 5 ], [ 3, 4 ], [ 3, 5 ], [ 4, 5 ] ], order := 10, 
  representatives := [ 1 ], schreierVector := [ -1, 2, 2, 1, 1, 1, 2, 1, 1, 1 
     ] )
gap> R:=GraphIsomorphismClassRepresentatives([A,B,ComplementGraph(A)]);;
gap> Length(R);
2
gap> List(R,VertexDegrees);
[ [ 6 ], [ 3 ] ]
gap> R:=GraphIsomorphismClassRepresentatives( 
>    [ rec(graph:=gamma, colourClasses:=[[1],[6],[2,3,4,5,7,8,9,10]]), 
>      rec(graph:=delta, colourClasses:=[[1],[6],[2,3,4,5,7,8,9,10]]), 
>      rec(graph:=ComplementGraph(gamma), colourClasses:=[[1],[6],[2,3,4,5,7,8,9,10]]) ] );; 
gap> Length(R);
3
\endexample

