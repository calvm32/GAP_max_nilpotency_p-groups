%Latex file manual.tex 17/1/95.
%Latest version for V2.4, 22/12/98.
%The style and layout was copied from the GAP manual.
\documentstyle{report}

%%%%%%%%%%%%%%%%%%%%%%%%%%%%%%%%%%%%%%%%%%%%%%%%%%%%%%%%%%%%%%%%%%%%%%%%%%%%%
%%
%%  The following commands intructs {\LaTeX} to stuff more on each  page  and
%%  to move each page towards to outer border.
%%
\topmargin -20mm
\textheight 24cm
\oddsidemargin  0cm
\evensidemargin  0cm
\textwidth 14cm


%%%%%%%%%%%%%%%%%%%%%%%%%%%%%%%%%%%%%%%%%%%%%%%%%%%%%%%%%%%%%%%%%%%%%%%%%%%%%
%%
%%  The following commands instruct  {\LaTeX}  to  separate the paragraphs in
%%  this manual with a small space
%%
\parskip 1.0ex plus 0.5ex minus 0.5ex
\parindent 0pt


%%%%%%%%%%%%%%%%%%%%%%%%%%%%%%%%%%%%%%%%%%%%%%%%%%%%%%%%%%%%%%%%%%%%%%%%%%%%%
%%
%%  'text'
%%
%%  'text' prints the text in  monospaced  typewriter  font.
%%  The text may contain all the usual characters  and |<name>| placeholders.
%%  |\'| can be used to enter a single  quote  character  into  the  text.
%%
\catcode`\'=13 \gdef'#1'{{\tt #1}}
\gdef\'{\char`'}


%%%%%%%%%%%%%%%%%%%%%%%%%%%%%%%%%%%%%%%%%%%%%%%%%%%%%%%%%%%%%%%%%%%%%%%%%%%%%
%%
%%  <text>
%%
%%  <text> prints  the text in  an italics font.
%%  The  text should not contain any special characters.  |\<| can be used to
%%  enter a less than character into the text.
%%
\catcode`\<=13 \gdef<#1>{{\it #1\/}}
\gdef\<{\char`<}


%%%%%%%%%%%%%%%%%%%%%%%%%%%%%%%%%%%%%%%%%%%%%%%%%%%%%%%%%%%%%%%%%%%%%%%%%%%%%
%%
%%  *text*
%%
%%  *text*  prints the text  in boldface font.
%%  The text may contain all  the usual characters.
%%  |\*| can be used to enter a star into the text.
%%
\catcode`\*=13 \gdef*#1*{{\bf #1}}
\gdef\*{\char`*}
\gdef\^{\char`^}


%%%%%%%%%%%%%%%%%%%%%%%%%%%%%%%%%%%%%%%%%%%%%%%%%%%%%%%%%%%%%%%%%%%%%%%%%%%%%
%%
%%  |text|
%%
%%  |text| prints the text between the two  pipe  symbols in typewriter style
%%  obeying the   linebreaks and spaces  in  the   manual.
%%  It should be used to  enter lengthy examples  into the text.
%%  If the hash character '\#' appears in the example the text between it
%%  and  the end of the line  is set in  ordinary mode,
%%  i.e., in  roman   font with  all the  possibilities ordinary available.
%%  |\|'\|' can be used to  enter  a  pipe symbol into the text.
%%
\catcode`\@=11

{\catcode`\ =\active\gdef\xvobeyspaces{\catcode`\ \active\let \xobeysp}}
\def\xobeysp{\leavevmode{} }

\catcode`\|=13
\gdef|{\leavevmode{}\hbox{}\begingroup
\def|{\endgroup}%
\catcode`\\=12\catcode`\{=12\catcode`\}=12\catcode`\$=12\catcode`\&=13
\catcode`\#=13\catcode`\^=12\catcode`\_=12\catcode`\ =12\catcode`\%=12
\catcode`\~=12\catcode`\'=12\catcode`\<=12\catcode`\"=12\catcode`\|=13
\catcode`\*=12\catcode`\:=12
\leftskip\@totalleftmargin\rightskip\z@
\parindent\z@\parfillskip\@flushglue\parskip\z@
\@tempswafalse\def\par{\if@tempswa\hbox{}\fi\@tempswatrue\@@par}%
\tt\obeylines\frenchspacing\xvobeyspaces\samepage}

\catcode`\@=12

{\catcode`\#=13
\gdef#{\begingroup
\catcode`\\=0 \catcode`\{=1 \catcode`\}=2 \catcode`\$=3 \catcode`\&=4
\catcode`\#=6 \catcode`\^=7 \catcode`\_=8 \catcode`\ =10\catcode`\%=14
\catcode`\~=13\catcode`\'=13\catcode`\<=13\catcode`\"=13\catcode`\|=13
\catcode`\*=13\catcode`\:=13
\catcode`\^^M=12 \Comment}}
{\catcode`\^^M=12
\gdef\Comment#1^^M{\rm \# #1 \endgroup \Newline}}
{\obeylines
\gdef\Newline{
}}

{\catcode`\&=13
\gdef&{\#}}

\gdef\|{\char`|}


%%%%%%%%%%%%%%%%%%%%%%%%%%%%%%%%%%%%%%%%%%%%%%%%%%%%%%%%%%%%%%%%%%%%%%%%%%%%%
%%
%%  <item>: <text>
%%
%%  This formats the  paragraph  <text>, i.e.,  everything between  the colon
%%  '\:' and the next  empty  line, indented 1 cm to the right in the manual.
%%  This
%%  convention should be  used to format  a list or an enumeration.    <item>
%%  should be  a single  word  or a short phrase.  It  may contain  all usual
%%  characters and the usual formatting stuff.  <text> is a  normal paragraph
%%  and may contain everything.   \:  can be used  to enter a colon character
%%  into the text.  As  an example consider the  following description.  This
%%  will print quite similar in the printed manual.
%%
%%      A group is represented by a record that must have the components
%%
%%      'generators': \\
%%              a list of group elements that  generate  the  group  that  is
%%              given by the group record.
%%
%%      'identity': \\
%%              the identity element of the group that is given by the  group
%%              record.
%%
\catcode`\:=13
\gdef:{\hangafter=1\hangindent=1cm\hspace{1cm}{}}
\gdef\:{\char`:}

%%%%%%%%%%%%%%%%%%%%%%%%%%%%%%%%%%%%%%%%%%%%%%%%%%%%%%%%%%%%%%%%%%%%%%%%%%%%%
%%
%%  "reference"
%%
%%  "reference" prints the  number of the  chapter or section  in the printed
%%  manual and is  displayed unchanged  in the  on-screen  documentation.  It
%%  should be used when referring to other  chapters or  sections.   The text
%%  should  not contain any special characters.  \"  can be  used  to enter a
%%  double quote into the text.
%%
\catcode`\"=13 \gdef"#1"{\ref{#1}}

\gdef\"{\char`"}

%%%%%%%%%%%%%%%%%%%%%%%%%%%%%%%%%%%%%%%%%%%%%%%%%%%%%%%%%%%%%%%%%%%%%%%%%%%%%
%%
%%  \GAP
%%
%%  \GAP can be used to enter the *sans serif* GAP logo  into  the  text.  If
%%  this is followed by spaces it should be enclosed in curly  braces  as  in
%%  |{\GAP}| is wonderful.
%%
\newcommand{\GAP}{{\sf GAP}}
\newcommand{\KBMAG}{{\sf KBMAG}}
\newcommand{\rkbp}{{\sf rkbp}}
\newcommand{\Automata}{{\sf Automata}}
\newcommand{\Automate}{{\sf Automate}}
\newcommand{\Grail}{{\sf Grail}}

%%%%%%%%%%%%%%%%%%%%%%%%%%%%%%%%%%%%%%%%%%%%%%%%%%%%%%%%%%%%%%%%%%%%%%%%%%%%%
%%
%%  \Chapter{<name>}
%%  \Section{<name>}
%%
%%  |\Chapter| and |\Section| begin a  new  chapter  or  section.  They  work
%%  basically like the ordinary |\chapter| and |\section| macros except  that
%%  they also create a label for <name>
%%
\catcode`\@=11 \catcode`\%=12 \catcode`\~=14

\newcommand{\Chapter}[1]{{\chapter{#1}~
\label{#1}}}

\newcommand{\Section}[1]{{\pagebreak[3]\section{#1}~
\label{#1}}}

\catcode`\@=12 \catcode`\%=14 \catcode`\~=13

%%%%%%%%%%%%%%%%%%%%%%%%%%%%%%%%%%%%%%%%%%%%%%%%%%%%%%%%%%%%%%%%%%%%%%%%%%%%%
%%
%%  make the title page
%%
\begin{document}

\title{KBMAG \\
Knuth--Bendix in Monoids, and Automatic Groups \\
Version 2.4}
\author{
Derek Holt \\
Mathematics Institute, University of Warwick, Coventry CV4 7AL, UK }
\date{22 December 1998}

\maketitle

%%%%%%%%%%%%%%%%%%%%%%%%%%%%%%%%%%%%%%%%%%%%%%%%%%%%%%%%%%%%%%%%%%%%%%%%%%%%%
%%
%%  include the preface
%%
%%%%%%%%%%%%%%%%%%%%%%%%%%%%%%%%%%%%%%%%%%%%%%%%%%%%%%%%%%%%%%%%%%%%%%%%%%
%%
%W  grpconst.tex          GrpConst documentation             Bettina Eick
%%

%%%%%%%%%%%%%%%%%%%%%%%%%%%%%%%%%%%%%%%%%%%%%%%%%%%%%%%%%%%%%%%%%%%%%%%%%
\Chapter{Preface}

The determination of all groups of a given order up to isomorphism
is a central problem in finite group theory. It has been initiated
in 1854 by A. Cayley who constructed the groups of order 4 and 6.

A large number of publications followed Cayley's work. For example,
Hall and Senior determined the groups of order $2^n$ for $n \leq 6$,
Neub{\accent127 u}ser listed the groups of order at most 100 except 64 and 96
and Laue added the groups of order 96, see \cite{HS64}, \cite{Neu67}
and \cite{Lau82}. These determinations partially relied on the
help of computers, but a general algorithm to construct groups had
not been used. The resulting catalogue of groups of order at most
100 has been available in GAP 3.

Then Newman and O'Brien introduced an algorithm to determine
groups of prime-power order, see \cite{OBr90}. An implementation 
of this method is available in the ANUPQ share package of GAP. 
This method has been used to compute the groups of order $2^n$ for 
$n \leq 8$ and the groups of order $3^n$ for $n \leq 6$, see
\cite{OBr88}, and the resulting groups are available in GAP.
Moreover, the large number of groups of order $2^8$ shows that
algorithmic methods are the only sensible way for group determinations
in this range.

In this share package we introduce practical methods to determine
up to isomorphism all groups of a given order. The algorithms 
are described in \cite{BE99}. These methods have been used to 
construct the non-nilpotent groups of order at most 1000, see 
\cite{BE1000}. The resulting catalogue of groups is available within
the small groups library of GAP 4. 

Our methods are not limited to groups of order at most 1000
and thus may be used to determine all or certain groups of 
higher order as well. However, it is not easy to say for 
which orders our methods are still practical and for which
not. As a rule of thumb one can say that the number of 
primes and the size of the prime-powers contained in the 
factorisation of the given order determine the practicability 
of the algorithm; that is, the more primes are contained in
the factorisation the more difficult the determination gets. 

As an example, the construction of all non-nilpotent groups of order 
$192 = 2^6 \cdot 3$ takes 17 minutes on an PC 400 Mhz. This is a medium 
sized application of our methods.  However, the construction of the groups
of order $768 = 2^8 \cdot 3$ takes already rather long (a few days) and 
can be considered as a limit of our methods. On the other hand, the 
groups of order $5425 = 5^2 \cdot 7 \cdot 31$ can be determined in 5 sec.
Moreover, if the determination of groups is restricted to groups
with certain properties, then this might increase the efficiency
of the construction process considerably. We include some example
applications of our methods to illustrate this at the end of the
manual.

Finally, we mention that the correctness of our algorithms is very
hard to check for a user; in particular, since there are no other
algorithms for the same purpose available, it might be difficult to
verify that our methods compute all desired groups. Thus we note here 
that methods implemented in this share package have been used to compute 
large parts of the Small Groups library and this, in turn, has been 
checked by the authors as described in \cite{BE99} and \cite{BE1000}.

Comments and suggestions on this share package are very welcome.
Please send them to 

\centerline{ beick@tu-bs.de or hubesche@tu-bs.de.}

Bug reports should also be e-mailed to either of these addresses.




%%%%%%%%%%%%%%%%%%%%%%%%%%%%%%%%%%%%%%%%%%%%%%%%%%%%%%%%%%%%%%%%%%%%%%%%%%%%%
%%
%%  make the table of contents
%%
\tableofcontents


%%%%%%%%%%%%%%%%%%%%%%%%%%%%%%%%%%%%%%%%%%%%%%%%%%%%%%%%%%%%%%%%%%%%%%%%%%%%%
%%
%%  and now the chapters
%%


\Chapter{Introduction}

This package provides \GAP-functions to reproduce the experimental
results described in our paper \cite{ER09}.

More precisely, it provides
\beginlist
\item{$\bullet$} functions to determine the rank, width and obliquity
  of a finite $p$-group,
\item{$\bullet$} functions to investigate the graph of all finite $p$-groups of a
  given rank, width and obliquity using the ANUPQ-package \cite{ANU06}, and
\item{$\bullet$} a library of finite quotients of certain infinite pro-$p$-groups
  of finite rank, width and obliquity.
\endlist

%%File kbm.tex
\Chapter{The Knuth--Bendix Program for Monoids}
\Section{kbprog}

'kbprog  [-r] [-ro] [-t <tidyint>] [-me <maxeqns>] [-ms <maxstates>]'\\
'| || || || || || || |[-mrl <maxreducelen>] [-mlr <maxlenleft> <maxlenright>]\
[-mo <maxoverlaplen>]'\\
'| || || || || || || |[-sort <maxoplen>] [-v] [-ve] [-silent]\
[-rk <minlen> <mineqns>]'\\
'| || || || || || || |[-lex] [-rec] [-rtrec] [-cn <confnum>] <monoidname>'

The program 'kbprog' has zillions of options. Only those that are relevant
to its use as a stand-alone Knuth--Bendix program on monoids are listed here.
Those pertaining to its use as part of the the automatic groups package are
dealt with in Chapter "The Automatic Groups Package".

'kbprog' takes its input from the file <monoidname>, which should contain a
declaration of a record defining a rewriting-system, in the format described
in Chapter "Introduction". Output is to two files, <monoidname>'.kbprog' and
<monoidname>'.reduce'. The first contains an updated declaration of the
original rewriting-system, in which the |equations|\ field contains the
list of all reduction equations found so far. If the process has completed,
and the system is now confluent, then a new field |isConfluent| will
have appeared, and will be set equal to 'true'. In the equations,
the left hand side will always be greater than the right hand side in the
ordering on strings that is being used (see options below).
The second file contains a finite state automaton,
which can be used, together with the contents of the first file,
to reduce words in the monoid generators. This is done using the
program 'wordreduce' (see Section "wordreduce (Knuth--Bendix)").

If the system is confluent, then these reductions will be to the unique
minimal word under the ordering being used that is equal in the monoid to the
input word. We can therefore solve the word problem in the monoid. In this
case, the language of the automaton will be this set of minimal words.
If the monoid is finite, then its order can be determined by using the
program 'fsacount' (see Section "fsacount (Knuth--Bendix)").
In any case, the words accepted
can be enumerated up to a specified length with the program 'fsaenumerate'
(see Section "fsaenumerate (Knuth--Bendix)").

The Knuth--Bendix process will more often than not run forever by default,
and so some conditions have to be specified under which it will stop.
These take the form of limits that are placed on certain variables,
such as the number of reduction equations. 
These limits can be given values by the user, either by use of command-line
options, or with a field setting in the input file
(and the former takes priority in case of conflict).
Wherever possible, if the program halts because one of the limits is exceeded,
it will print a message informing the user what has happened,
and output its current set of equations and the reduction machine.

It is also possible to halt the program interactively at any stage, by
sending it a single interrupt signal (which usually means typing Control-C).
If you do this, then it will halt and output at the next convenient
opportunity, and so you may have to wait a while. If you send two interrupt
signals, then it will abort immediately without outputting.

*Options*\\
For most of the command-line options, the same effect can also be achieved by 
use of a corresponding field setting in the input file. In case of conflict,
the convention is that options set via the command-line override the setting in
the input file. Note, however,  that the two most complicated ordering options,
the weighted-lex ordering and the wreath-product ordering, can only
be sensibly set within the input file, because they require the generators
to be given weights and levels, respectively, in these two cases.
See the options '-wtlex' and '-wreath' below, for further details.
\begin{description}
\item[|-r |]
This means resume after a previous run in which the output set of equations
was not confluent. Input will be taken from <monoidname>|.kbprog| instead of
from <monoidname>. The output will go to the same place, so the old
<monoidname>|.kbprog| will be overwritten.
It is useful if the program halted on a previous run due to some limit
being exceeded, and you wish to resume with a higher limit.
\item[|-ro |]
This is similar to |-r|, but in addition to taking the 
input from <monoidname>|.kbprog|, the original equations, in the file
<monoidname>, will also be read in and re-inserted at the end of the list.
This is sometimes necessary or advisable, if on the previous run not all
equations have been output, or some have been rejected because they were too
long. In that situation, there is a danger that the monoid defined by
the equations may have changed, and it can always be reset to the original
by re-inserting the original equations.
The output will go to the usual place, so the old
<monoidname>|.kbprog| will be overwritten.
\item[|-t| <tidyint>] | |\newline
After finding <tidyint> new reduction equations, the program interrupts
the main process of looking for overlaps, to tidy up the existing set of
equations. This means eliminating any redundant equations and performing
some reductions on their left and right hand sides to make the set as
compact as possible. (The point is that equations discovered later often
make older equations redundant or too long.) The default value of
<tidyint> is 100, and it can be altered with this option. Different values
work better on different examples. This parameter can also be set by
including a |tidyint| field in the input file.
\item[|-me| <maxeqns>] | |\newline
This puts a limit on the number of reduction equations.
The default is 32767.
If exceeded, the program will stop and output the current equations.
It can also be set as the field 'maxeqns' in the input file.
\item[|-ms| <maxstates>] | |\newline
This is less important, and not usually needed.
It sets a limit on the number of states of the finite state automaton
used for word reduction.
If exceeded, the program will stop and output the current equations.
By default, there is no limit, and the space allocated is increased
dynamically as required. Occasionally, the space required can increase too fast
for the program to cope; in this case, you can try setting a higher limit.
It can also be set as the field 'maxstates' in the input file.
The space needed for the reduction automaton can also be restricted by
using the |-rk| (Rabin-Karp reduction) option - see below.
\item[|-mrl| <maxreducelen>] | |\newline
Again, this is not needed very often. It is the maximum allowed length that
a word can reach during reduction. By default it is 32767.
If exceeded, the program is forced to abort without outputting.
It is only likely to be exceeded if you are using a recursive ordering on words.
It can also be set as the field 'maxreducelen' in the input file.
\item[|-mo| <maxoverlaplen>] | |\newline
If this is used, then only overlaps of total length at most <maxoverlap>
are processed.
Of course this may cause the overlap search to complete on a set
of equations that is not confluent. If this happens, you can always resume
with a higher (or no) limit.
This parameter can also be set as the field 'maxoverlaplen' in the input file.
\item[|-mlr| <maxlenleft> <maxlenright>] | |\newline
If this is used, then only equations in which the left and right hand sides
have lengths at most <maxlenleft> and <maxlenright>, respectively, are
kept. Of course this may cause the overlap search to complete on a set
of equations that is not confluent. If this happens, you can always resume
with higher limits. In some examples, particularly those involving
collapse (i.e. a large intermediate set of equations, which later simplifies
to a small set), it can result in a confluent set being found much more
quickly. It is most often useful when using a recursive ordering on words. 
Another danger with this option is that sometimes discarding equations can
result in information being lost, and the monoid defined by the equations
changes. If this may have happened, a warning message will be printed at
the end. In this case, the recommended policy is to re-run with the 
|-ro| option. In later versions it will be possible to do this automatically.
This option can also be set as a field in the input file.
The syntax for this is

'maxstoredlen \:= [<maxlenleft>,<maxlenright>]'
\item[|-sort| <maxoplen>] | |\newline
This causes equations to be output in order of increasing length of their
left hand sides, rather than the default, which is to output them in the
order in which they were found. <maxoplen> should be a non-negative integer.
If it is positive, then only equations with left hand sides having length
at most <maxoplen> are output. If it is zero, then all equations are output.
Of course, if <maxoplen> is positive, there is a danger that the monoid
defined by the output equations may be different from the original.
In this case, the safest thing is to edit the output file,
re-insert the original relations and re-run (possibly with higher length
limits). In later versions it will be possible to do this automatically.
This option can also be set as fields in the input file.
The syntax for this is

'sorteqns \:= true, maxoplen \:= <maxoplen>'
\item[|-rk| <minlen> <mineqns>] | |\newline
Use the Rabin-Karp algorithm for word-reduction on words having length at least
<minlen>, provided that there are at least <mineqns> equations.
This uses less space than the default reduction automaton, but it is
distinctly slower, so it should only be used when you are seriously short of
memory.
In fact, if the program halts and outputs for any reason, then
the full reduction automaton is output as normal, so it is only really
useful for examples in which collapse occurs - i.e. at some intermediate
stage of the calculation there is a very large set of equations, which later
reduces to a much smaller confluent set. However, this situation is not
uncommon when analysing pathological presentations of finite groups, and
this is one situation where the performance of the Knuth-Bendix algorithm can
be superior to that of Todd-Coxeter coset enumeration.
The best settings for <minlen> and <mineqns> vary from example to
example - generally speaking, the smaller <minlen> is, the slower things
will be, so set it as high as possible subject to not running out of memory.
<mineqns> should be set higher than you expect the final number of
equations to be.
This option can also be set as a field in the input file.
The syntax for this is

'RabinKarp \:= [<minlen>,<mineqns>]'
\item[|-v |]
The verbose option. Regular reports on the current number of equations, etc. are
output. This is to be recommended for interactive use.
This parameter can also be set by including a |verbose| field in the input
file, and setting it equal to 'true'.
\item[|-ve |]
The verbose-equations option. In addition to the output produced under
|-v|, each new equation is printed as it is found, together with the
numbers of the two equations for which the overlap gave rise to this
new equation. This can be useful in small examples for reconstructing
the derivation of a particular equation in the monoid.
\item[|-silent|]
There is no output at all to 'stdout'. In particular, the reason for
halting will not be printed.
This parameter can also be set by including a |silent| field in the input
file, and setting it equal to 'true'.
\item[|-lex|]
Use the short-lex ordering on strings. This is the default ordering.
Shorter words come before longer, and for words of equal length,
lexicographical ordering is used, using the given ordering of the generators.
It can also be set as a field in the input file. The syntax for this is

|ordering := "shortlex"|
\item[|-rec, -rtrec|]
Use a recursive ordering on strings. 
There are various ways to define this. Perhaps the quickest is as
follows. Let $u$ and $v$ be strings in the generators.
If one of $u$ and $v$, say $v$,  is empty, then $u \ge v$.
Otherwise, let $u=u^\prime a$ and $v=v^\prime b$,
where $a$ and $b$ are generators.
Then $u > v$ if and only if one of the following holds\:
\begin {description}
\item[(i)] $a = b$ and $u^\prime > v^\prime$;
\item[(ii)] $a > b$ and $u > v^\prime$;
\item[(ii)] $b > a$ and $u^\prime > v$.
\end {description}
This is the ordering used for the |-rec| option. The |-rtrec| option is
similar, but with $u=au^\prime$ and $v=bv^\prime$;
occasionally one or the other runs
significantly quicker, but usually they perform similarly.
It can also be set as a field in the input file. The syntax for this is

|ordering := "recursive"| \hspace{1cm} or  \hspace{1cm}
|ordering := "rt_recursive"|

\item[|-wtlex|]
Use a weighted-lex ordering.
Although this option does exist as a command-line option, it will usually
be specified within the input file, because each generator needs to be
assigned a weight, which should be a non-negative integer. The \'\'length\"
of words in the generators is then computed by adding up the weights of the
generators in the words. Otherwise, ordering is as for short-lex.
An example of assignments within the the input file is:

|ordering := "wtlex", weight := [2,1,6,3,0],|

which assigns weights 2,1,6,3 and 0 to the generators. The length of the
list of weights must be equal to the number of generators. The assignment
of the 'weight' field must come after the 'generatorOrder' field.
\item[|-wreath|]
Use a wreath-product ordering.
Although this option does exist as a command-line option, it will usually
be specified within the input file, because each generator needs to be
assigned a level, which should be a non-negative integer.
In this ordering, two strings involving generators of the same level are
ordered using short-lex, but all strings in generators of a higher level are
larger than those involving generators of a lower level. That is not a
complete definition; one can be found  on pages 46 -- 50 of \cite{Sims94}.
Note that the recursive ordering is the special case in which the level
of generator number $i$ is $i$.
An example of assignments within the the input file is:

|ordering := "wreathprod", level := [4,3,2,1],|

which assigns levels 4,3,2 and 1 to the generators. The length of the
list of levels must be equal to the number of generators. The assignment
of the 'level' field must come after the 'generatorOrder' field.
\item[|-cn| <confnum>] | |\newline
If <confnum> overlaps are processed and no new equations are discovered, then
the overlap searching process is interrupted, and a fast check for
confluence performed on the existing set of equations.
The default value is 500. Doing this too often wastes time, but doing it
at the right moment can also save a lot of time. Sometimes a particular
value works very well for a particular example, but it is difficult
to predict this in advance! If <confnum> is set to 0, then the fast
confluence check is performed only when the search for overlaps is
complete.
It can also be set as the field 'confnum' in the input file.
\end{description}

*Exit status*\\
The exit status is 0 if 'kbprog' completes with a confluent set of equations,
2 if it halts and outputs a non-confluent set because some limit has
been exceeded, or it has received an interrupt signal, and 1 if it exits
without output, with an error message.

\Section{wordreduce (Knuth--Bendix)}
'wordreduce  [-kbprog/-diff1/-diff2/-diffc] [-mrl <maxreducelen>] monoidname' 

This program can be used either on the output of 'kbprog' or on the output
of the automata package. The '[-diff<x>]' options refer to the latter use,
which is described in Section
"wordreduce (automatic groups)".
The former use is the default if
the file <monoidname>'.kbprog' is present; to be certain, call the
'-kbprog' flag explicitly.

'wordreduce' reduces words using the output of a run of 'kbprog', which is
read from the files <monoidname>'.kbprog' and <monoidname>'.reduce'.
The reductions will always be
correct in the sense that the output word will represent the same group
element as the input word. If the system of equations in <monoidname>'.kbprog'
is confluent, then the reduction will be to the minimal word that represents
the group element under the ordering on strings of generators that was used
by 'kbprog'. It can therefore be used to solve the word problem in the
monoid. If the system is not confluent, then there will be some pairs of words 
which are equal in the monoid, but which reduce to distinct words, and
so this program cannot be used to solve the word problem.
'wordreduce' prompts for the words to be input at the terminal.

The option '-mrl\ <maxreducelen>' puts a maximum length on the words to
be reduced. The default is 32767.

\Section{fsacount (Knuth--Bendix)}

'fsacount  [-ip d/s] [-silent] [-v] [<filename>]'

This is one of the finite state automata functions. See Chapter 
"Programs for Manipulating Finite State Automata" for the complete list.
The size of the accepted language is counted, and the answer (which may
of course be infinite) is output to 'stdout'. Input is from <filename> if
the optional argument is present, and otherwise from 'stdin', and it
should be a declaration of a finite state automaton record.
If 'kbprog' outputs a confluent set of equations for the monoid in the file
<monoidname>, then running 'fsacount <monoidname>.reduce'
will calculate the size of the monoid.
For description of options, see Section
"Exit Status of Programs and Meanings of Some Options".

\Section{fsaenumerate (Knuth--Bendix)}

'fsaenumerate  [-ip d/s] [-bfs/-dfs] <min> <max> [<filename>]' 

This is one of the finite state automata functions. See Chapter 
"Programs for Manipulating Finite State Automata" for the complete list.
Input is from <filename> if
the optional argument is present, and otherwise from 'stdin', and it
should be a declaration of a finite state automaton record.
<min> and <max> should be non-negative integers with <min> $\le$ <max>.
The words in the accepted language having lengths at least <min> and at
most <max> are enumerated, and output as a list of words to
'stdout' or to the file <filename>'.enumerate'.
If 'kbprog' outputs a confluent set of equations for the monoid in the file
<monoidname>, then running 'fsaenumerate <min> <max> <monoidname>.reduce'
will produce a list of elements in the monoid of which the reduced 
word representatives have lengths between <min> and <max>.
If the option '-dfs' (depth-first search - the default) is called,
then the words
in the list will be in lexicographical order, whereas with '-bfs'
(breadth-first-search), they will be in order of increasing length, and in
lexicographical order for each individual length (i.e. in short-lex order).
Depth-first-search is marginally quicker.
For description of other options, see Section
"Exit Status of Programs and Meanings of Some Options".
\Section{Examples (Knuth--Bendix)}

In this section, we mention some of the examples in the 'kb\_data' directory.
These can usefully be used as test examples, and some of them have been
included to demonstrate particular features.

The 'degen' examples  are all of the trivial group. Note, in particular,
'degen4a', 'degen4b' and 'degen4c'. These are the first three of an
infinite sequences of increasingly complicated presentations of the
trivial group, due to B.H. Neumann. 'kbprog' will run quite quickly on
'degen4b' (although no current Todd-Coxeter program will complete on this
presentation), but it does not appear to complete on 'degen4c'.

The example 'ab2' is the free abelian group on two generators.
It terminates with a confluent set for the given ordering of the
generators, |[a,A,b,B]|, but does not terminate with the ordering |[a,b,A,B]|.

Several of the examples are of finite groups. 
Others are monoid presentations, where generators are not supplied with
inverses. Try 'f25monoid', which is the presentation of the Fibonacci group
$F(2,5)$, but as a monoid. In fact, the structure is almost identical to
the group in this example. The group is cyclic of order 11.
The monoid has order 12, the extra element being the empty word.
The corresponding semigroup (without the empty word) is isomorphic to the
group.

In the examples 'nilp2', 'nonhopf', 'heinnilp' and 'verfiynilp', the
'recursive' ordering is essential. The last two of these are examples of
the use of Knuth--Bendix to prove that a presentation defines a nilpotent
group (first proposed by Sims). In 'verifynilp', things are made much
easier by using the 'maxstoredlen' parameter (or equivalently the '-mrl'
option). (Appropriate settings are already in the input file.)

The example 'f27monoid' is a monoid presentation
corresponding to the Fibonacci group $F(2,7)$, which has order 29.  As is
the case with $F(2,5)$, the monoid is the same structure with the empty
word thrown in, but this example is rather more difficult for 'kbprog'.
The best approach is to use a recursive ordering with a limit on the
lengths of equations stored ('-mrl 15 15' works well for 'f27monoid').
This will halt with the warning message that information may have been lost.
The original equations should then be adjoined to the output equations,
which is achieved by re-running 'kbprog' with the '-ro' option,
(and no limits on lengths).
It will then quickly complete with a confluent set. This is typical of a
number of difficult examples, where good results can be obtained by running
more than once.

%%File autgp.tex
\Chapter{The Automatic Groups Package}

The main aim of this package is to calculate the finite state automata
associated with a short-lex automatic group. At the end of a
successful calculation, four automata will be stored. These are the first
and second word-difference machines, the word-acceptor, and the multiplier.
The descriptions in this  chapter of the manual will assume some familiarity
with these objects on the part of the reader.
See \cite{EHR91} or \cite{Holt94} for details. One difference in the
current version from the existing {\Automata} package
is that a single multiplier automaton is calculated,
rather than a separate one for each generator.
This single multiplier is known as the general multiplier. Some of its states
are labeled by one of the group generators. These states are the accepting
states for that generator. To obtain the individual multipliers in minimized
form, run the program 'gpmult' (see Section "gpmult"). They are not normally
significantly smaller than the general multiplier, and there is one of them
for each generator, so usually the general multiplier provides a much more
compact way of storing the same information than the individual multipliers.

The resulting automata
can be used for reducing words to their unique minimal representative in
the group under the short-lex ordering (and thereby enabling the
word-problem to be solved in the group), and also for counting and
enumerating the accepted language.
See Sections "wordreduce (automatic groups)", "fsacount (automatic groups)"
and "fsaenumerate (automatic groups)" for details.
(In fact the multiplier automaton is not needed for any of these processes,
but it forms part of the automatic structure of the group from a
theoretical viewpoint, so we regard it as part of the output of the
calculation.)

There is also a program 'gpgeowa' that attempts to construct the geodesic
word acceptor of a word-hyperbolic group. In fact, by a result of
Papasoglu (\cite{PAP}), when it succeeds, it verifies that the
group is  word-hyperbolic. This program described in Section "gpgeowa".

There are two possibilities for computing the automatic structure.
The simplest is to use the program 'autgroup', which is in fact a
Unix Bourne-shell script, which runs the three C-programs involved,
and attempts to make a sensible choice of options.
The other possibility is to run these three programs 'kbprog -wd',
'gpmakefsa' and 'gpaxioms' individually, and to select the options oneself.
'kbprog -wd' runs the Knuth--Bendix process on the defining relations of the
group, and calculates the resulting word-differences arising from the
equations generated. If the group is short-lex automatic, then the set
of word-differences is finite, and so the calculated set will eventually
be complete; however, 'kbprog -wd' itself will normally run forever, generating
infinitely many equations, and so the difficulty is to devise sensible
halting criteria. Clearly one wants to halt as soon as all of the
word-differences have been found,if possible. Further details are given in
Section "kbprog -wd". 'gpmakefsa' uses the word-difference output by
'kbprog -wd' to compute the word-acceptor and multiplier automata.
It performs some checks on these which can reveal if the set of
word-differences used was not in fact complete, and find some of the
missing ones. It can then try again with the extended sets of
word-differences. Finally, the program 'gpaxioms' performs the
axiom-checking process, which proves the correctness of the automata
calculated. This process is expensive on resources in terms of both time
and space. Interestingly enough, we have never known it to complete and
return a negative answer; the reason for this is that the tests carried
out in 'gpmakefsa' tend to detect any errors in the automata.
So please inform me immediately if 'gpaxioms' ever reports that the
the structure calculated by 'kbprog\ -wd' and 'gpmakefsa' is incorrect!

For simple examples, these programs work quickly, and do not require
much space. For more difficult examples, they are often capable of
completing successfully, but they can sometimes be expensive in terms of
time and space requirements. If you are running out of space on your computer,
or it is starting to swap heavily, then there are some options available
in some cases (such as '-ip s' and '-f' for 'gpmakefsa' and 'gpaxioms')
which will cause things to use less space, but to take longer.
Another point to be borne in mind is that they produce temporary disk
files which the user does not normally see (because they are
automatically removed after use), but can occasionally be very large.
If you are in danger of exceeding your filestore allocation, or filling up
your disk partition, then you might try running the programs in the '/tmp'
directory. In any case, if a program halts unnaturally (perhaps because
you interrupt it), then you must remove these temporary files yourself.
If the file containing the original group presentation is  named
<groupname>, then all files created by the programs will have names of
form <groupname>'.'<suffix>.
\Section{autgroup}

'autgroup  [-l] [-v] [-silent] [-diff1] [-f] <groupname>'

Compute the finite state automata that constitute the automatic structure
of a short-lex automatic group. Input is taken from the file <groupname>,
which should contain a declaration of a record defining a rewriting-system, in
the format described in Chapter "Introduction". The rewriting-system must
define a group, so all monoid generators must have inverses listed explicitly
(there is no default convention for names of inverses). Output is to the
four files <groupname>'.diff1', <groupname>'.diff2', <groupname>'.wa' and
<groupname>'.gm', which contain, respectively, the first and second
word-difference automata, the word acceptor and the general multiplier.

*Options*\\
For a general description of the options '[-l]', '[-v]' and '[-silent]',
see Section
"Exit Status of Programs and Meanings of Some Options".
For greater flexibility in choice of options, run the programs 'kbprog\ -wd',
'gpmakefsa' and 'gpaxioms' individually, rather than using 'autgroup'. 

\begin{description}
\item[|-l |]
This causes 'gpmakefsa' and 'gpaxioms' to be run with the '-l' option, which
means large hash-tables. However, '-l' also causes 'kbprog\ -wd' to be run
with larger parameters, and you are advised to use it only after you
have tried first without it, since it will cause easy examples to take
much longer to run.
\item[|-diff1|]
This causes 'gpmakefsa' to be run with the '-diff1' option. See Section
"gpmakefsa" for further details.
\item[|-f |] This causes 'gpmakefsa' and 'gpaxioms' to be run with the
'-f' and '-ip s' options. See Sections "gpmakefsa" and "gpaxioms" for
further details. This is the option to choose if you need to save space,
and don\'t mind waiting a bit longer.
\end{description}
\Section{kbprog -wd}
'kbprog -wd [-t <tidyint>] [-me <maxeqns>] [-ms <maxstates>] \
[-mwd <maxwdiffs>]'\\
'| || || || || || || |[-hf <halting\_factor>] [-mt <min\_time>] \
[-rk <minlen> <mineqns>]'\\
'| || || || || || || |[-v] [-vwd] [-silent] [-cn <confnum>] <groupname>'

Some of these options are of course the same as for when 'kbprog' is being
used as a stand-alone Knuth--Bendix program -- see Section "kbprog".

'kbprog\ -wd' takes its input from the file <groupname>, which should contain a
declaration of a record defining a rewriting-system, in the format described
in Chapter "Introduction". Only the default ordering |"shortlex"| may be used,
since the other orderings make no sense in the automatic groups context.
Remember that the list of generators must include the inverses of all
generators.  Since the rewriting-system must define a group, rather than just
a monoid, the list of inverses, as specified in the 'inverses' field, must be
complete and involutory; i.e. all monoid generators must have inverses, and if
the inverse of <g1> is <g2> then the inverse of <g2> must be <g1>.

The Knuth--Bendix completion process is carried out on the group presentation.
However, unlike in the standalone usage of 'kbprog', the extended set of
equations and the reduction automaton are not output. Instead, the two
word-difference automata arising from the equations are output when the program
halts. These are printed to the two files <groupname>'.diff1' and
<groupname>'.diff2'. The difference between them is that the first automaton
contains only those word-differences and transitions that arise from the
equations themselves. For the second automaton, the set of word-differences
arising from the equations is closed under inversion, and all possible
transitions are included. So the second automaton has both more states and
generally more transitions per state than the first. For their uses, see
Section "gpmakefsa" below.

The main problem with this program is that, unless it completes with a
confluent set of equations, which happens relatively rarely for infinite groups,
the process will continue indefinitely, and so halting criteria have to be
chosen. If the group really is short-lex automatic with this choice of
ordered generating set, then the set of word-differences will be finite.
The general idea is to wait until it appears to have become constant, and
then stop.

There are two ways to do this. The first is to run 'kbprog\ -wd'
interactively with the '-v' option, when regular reports will be printed
on the current number of word-differences. (The word-differences are
calculated after each tidying operation on the equations; see the
'-t <tidyint>' option below.) The program can be interrupted at any time by the
user by sending it a single interrupt signal (which usually means typing
Control-C).  If you do this, then it will halt and output the current 
word-difference machines at the next convenient
opportunity (and so you may have to wait a while). If you send two interrupt
signals, then it will abort immediately without outputting.
The second way (and the one used by the shell-script 'autgroup') is to call
options which cause the program to halt automatically when the number of
word-differences has not increased for some time. This method is described
below under the option descriptions  '-hf <halting\_factor>' and
'-mt <min\_time>'.

This works well in simple examples, but there are various problems
associated with the more difficult examples. Firstly, it can happen that
most word-differences are found relatively quickly, but a few take much longer
to appear, and so the program is halted too early (either interactively or
automatically). This does not always matter, because the next program in
the sequence 'gpmakefsa' has the potential for finding missing word-differences
by performing checks on the automata that it calculates. However, this
will inevitably make 'gpmakefsa' run slower and use more space, and if too
many word-differences are missing, then it might not succeed at all. The thing
to do then, is to give up, and try running 'kbprog\ -wd' for longer, with more
stringent halting criteria (and possibly with a larger value of <tidyint>).
Another problem (the opposite problem) is that in some examples spurious
word-differences keep appearing and disappearing again, and so all required
word-differences may have been found long ago, but the reported number is
showing no sign of becoming constant. The only thing to do
when the number of word-differences keeps increasing and later decreasing
dramatically, is to try stopping it anyway (when the number is at
a low point), and then run 'gpmakefsa'.

Finally, if the number of word-differences is observed to increase quickly and
steadily, and gets up to about 2000, then it is likely that the group is not
short-lex automatic with this choice of ordered generators (and even if it is,
the size of the automata involved is likely to be too large for the
programs). Since short-lex automaticity is dependent on the choice of the
ordered generating set in some examples, it is worth trying different
choices of generators for the same group, and possibly different orderings
of the generators.

*Options*
\begin{description}
\item[|-t| <tidyint>] | |\newline
After finding <tidyint> new reduction equations, the program interrupts
the main process of looking for overlaps, to tidy up the existing set of
equations. This means eliminating any redundant equations and performing
some reductions on their left and right hand sides to make the set as
compact as possible. (The point is that equations discovered later often
make older equations redundant or too long.) The default value of
<tidyint> is 100, and it can be altered with this option. Different values
work better on different examples. This parameter can also be set by
including a |tidyint| field in the input file.
The word-differences arising from the equations are calculated
after each such tidying and the number reported if the '-v' option is called.
The best strategy in general is to try a small value of <tidyint> first and,
if that is not successful, try increasing it. Large values such as 1000 work
best in really difficult examples.
\item[|-me| <maxeqns>] | |\newline
This puts a limit on the number of reduction equations.
The default is 32767.
If exceeded, the program will stop and output the word-difference
automata arising from the current equations.
It can also be set as the field 'maxeqns' in the input file.
\item[|-ms| <maxstates>] | |\newline
This is less important, and not usually needed.
It sets a limit on the number of states of the finite state automaton
used for word reduction.
If exceeded, the program will stop and output the 
word-difference automata arising from current equations.
By default, there is no limit, and the space allocated is increased
dynamically as required. Occasionally, the space required can increase too fast
for the program to cope; in this case, you can try setting a higher limit.
It can also be set as the field 'maxstates' in the input file.
\item[|-mwd| <maxworddiffs>] | |\newline
Again this is not needed very often. It puts a bound on the number of
word-differences allowed. Normally, it is increased dynamically as required,
and so it does not need setting. Occasionally, it increases too fast for
the program to cope, and then it has to abort without output. If this
happens, try a larger setting.
\item[|-hf| <halting\_factor>]
\item[|-mt| <min\_time>]| |\newline
These are the experimental halting-options. If '-hf' is called, then
<halting\_factor> should be a positive integer, and represents a percentage.
After each tidying, it is checked whether both the number of
equations and the number of states have increased by more than
<halting\_factor> percent since the number of word-differences was last
less than what it is now.
If so, then the program halts and outputs the
word-difference automata arising from the current equations. A sensible value
seems to be 100, but occasionally a larger value is necessary. If the
'-mt' option is also called, then halting only occurs if at least <min\_time>
seconds of cpu-time have elapsed altogether.
This is sometimes necessary to prevent very early premature halting.
It is not very satisfactory, because of course the cpu-time
depends heavily on the particular computer being used, but no reasonable
alternative has been found yet.
There is no point in calling '-mt' without '-hf'.
\item[|-rk| <minlen> <mineqns>] | |\newline
Use the Rabin-Karp algorithm for word-reduction on words having length at least
<minlen>, provided that there are at least <mineqns> equations.
This uses less space than the default reduction automaton, but it is
distinctly slower, so it should only be used when you are seriously short of
memory.
The best settings for <minlen> and <mineqns> vary from example to
example - generally speaking, the smaller <minlen> is, the slower things
will be, so set it as high as possible subject to not running out of memory.
This option can also be set as a field in the input file.
The syntax for this is

'RabinKarp \:= [<minlen>,<mineqns>]'
\item[|-v |]
The verbose option. Regular reports on the current number of equations, etc. are
output. This is to be recommended for interactive use.
This parameter can also be set by including a |verbose| field in the input
file, and setting it equal to 'true'.
\item[|-vwd |]
The verbose word-differences option. In addition to the output produced
under |-v|, each new word-difference found is printed to 'stdout', together
with the equation from which it was calculated.
\item[|-silent|]
There is no output at all to 'stdout'.
This parameter can also be set by including a |silent| field in the input
file, and setting it equal to 'true'.
\item[|-cn| <confnum>] | |\newline
If <confnum> overlaps are processed and no new equations are discovered, then
the overlap searching process is interrupted, and a fast check for
confluence performed on the existing set of equations.
Setting <confnum> equal to 0 turns this feature off completely.
If, as is often the case, you are quite certain that the process is not going
to halt at all, then you should set <confnum> to 0, since the confluence
tests will merely waste time. (In fact, this should arguably be the
default setting for 'kbprog\ -wd'.)
It can also be set as the field 'confnum' in the input file.
\end{description}

*Exit status*\\
The exit status is 0 if 'kbprog' completes with a confluent set of equations,
or if it halts because the condition in the '-hf <halting\_factor>' option
has been fulfilled,
2 if it halts and outputs because some other limit has
been exceeded, or it has received an interrupt signal, and 1 if it exits
without output, with an error message.
This information is used by the shell-script 'autgroup'.

\Section{gpmakefsa}

'gpmakefsa  [-diff1/-diff2] [-me <maxeqns>] [-mwd <maxwdiffs>] [-f] [-l]'\\
'| || || || || || || || || || |[-ip d/s[dr]] [-opwa d/s] [-opgm d/s]\
[-v] [-silent] <groupname>'

Construct the word-acceptor and general multiplier finite state automata
for the short-lex automatic group, of which the rewriting-system defining
the group is in the file <groupname>.
It assumes that 'kbprog\ -wd' has already been run on the group.
Input is from <groupname>'.diff2' and, if
the '-diff1' option is called, also from <groupname>'.diff1'. The
word-acceptor is output to <groupname>'.wa' and the general multiplier to
<groupname>'.gm'. Certain correctness tests are carried out on the
constructed automata. If they fail, then new word-differences will be
found and <groupname>'.diff2' (and possibly also <groupname>'.diff1')
will be updated, and the multiplier (and possibly the word-acceptor)
will then be re-calculated.

There are programs available which construct the two automata individually,
and also one which performs the correctness test, but it is unlikely that
you will want to use them. They are mentioned briefly in
Section "Other programs".

The algorithm is slightly different according to whether or not
<groupname>'.diff1' or <groupname>'.diff2' is used as input to construct
the word-acceptor. This is controlled by the '-diff1' and '-diff2' options.
The latter is the default; i.e. the default is to use <groupname>'.diff2'.
Theoretically, <groupname>'.diff1' should work, and indeed it is more
efficient when it does work, because the first word-difference automaton is
always smaller than the second. However, in many examples it turns out that
<groupname>'.diff1' as output by 'kbprog\ -wd' is incorrect. In this case, it
is nearly always more efficient overall to use <groupname>'.diff2', which has
a much higher chance of calculating the correct word-acceptor first time. We
have therefore chosen to make this the default. There are one or two
examples, however, in which the use of <groupname>'.diff2' causes severe
space problems in constructing the word-acceptor, whereas <groupname>'.diff1'
does not. If you observe this to be the case, then try again with the
'-diff1' option set.

In the default setting, the word-acceptor and general multiplier are both
constructed using <groupname>'.diff2'. A check is then carried out on the
multiplier. In fact, it is checked that, for every word $u$ accepted by
the word-acceptor, and every generator $a$, there exists a word $v$
accepted by the word-acceptor such that $(u,v)$ is accepted by the multiplier,
where $ua$ and $v$ are equal in the group.
If this test fails, then the multiplier is incorrect, and a number of explicit
words $u$ and generators $a$ are calculated for which the test fails.
These give rise to equations $(u,v)$ (where $v$ is the word to which
$ua$ reduces), which produce new word-differences, which are in turn used
to re-calculate the second word-difference machine. The file
<groupname>'.diff2' is then updated, and the multiplier recalculated. It
may turn out in the course of this calculation that the word-acceptor also
needs recalculating, which is done if necessary. This process continues
until the multiplier passes the test.

Under the '-diff1' option, the file <groupname>'.diff1' is used to construct
the word-acceptor, but <groupname>'.diff2' is still used for the multiplier.
It may happen, during construction of the multiplier, that some equations
are found that should be accepted by the first word-difference machine, but
are not. These are used to correct the first word-difference machine, and
the file <groupname>'.diff1' is updated. Otherwise, things proceed as for the
'-diff2' option.

*Other options*
\begin{description}
\item[|-me| <maxeqns>] | |\newline
This specifies an upper limit on the number of equations that are processed
when correcting the first or second word-difference machines, as described
above. The default is 512. If it is too small, then the main testing and
correcting process on the multipliers may have to be repeated many more times
than necessary, but if it is too big, then the process of updating the
word-difference machines can be very slow.
\item[|-mwd| <maxwdiffs>] | |\newline
This puts an upper limit on the number of word-differences allowed after the
correction. It is rarely necessary to call this. If it is necessary, then the
program will halt with an informative error message.
\item[|-f |]
This is an option which saves space, at the expense of slightly increased
cpu-time. The multiplier automaton as initially constructed can be very
large. It is then minimized. Under this option, the unminimized multiplier
is not read in all at once, but kept in a file, and read in line-by-line
during minimization.
\end{description}
The remaining options are standard, and described in Section
"Exit Status of Programs and Meanings of Some Options".
Note, however, that the two '-op' options,
'-opwa' and '-opgm', refer to the output format of the word-acceptor and
general multiplier, resepctively. Since the former is one-variable and
the latter two-variable, the default is dense for the former and sparse
for the latter.
\Section{gpaxioms}

'gpaxioms [-ip d/s[dr]] [-op d/s] [-silent] [-v] [-l] [-f] [-n] <groupname>'

This program performs the axiom-checking process on the word-acceptor and
general multiplier of the short-lex automatic group,
of which the rewriting-system defining the group is in the file <groupname>.
It assumes that 'kbprog\ -wd' and 'gpmakefsa' has already been run on the
group.
It takes its input from <groupname> and <groupname>'.gm'. 
There is no output to files (although plenty of intermediate files are
created and later removed during the course of the calculation).

If successful, there will be no output apart from routine reports, and the exit
status will be 0. If unsuccessful, a message will be printed to 'stdout'
reporting that a relation test failed for one of the group relations, and
the exit status will be 2. I have never known this to happen, at least on
the output of a successful run of 'gpmakefsa', so if it does, please inform
me immediately!

*Options*
\begin{description}
\item[|-f|]
This is the same as in 'gpmakefsa'. That is, unminimized automata are not
read into memory all at once during minimization.
\item[|-n|]
Normally, 'gpaxioms' works by first calculating a general multiplier for
all words of length two in the group generators, and using these to
build up the composite multipliers for longer words. This seems to be
a good policy for nearly all examples, except possibly when there is
a very large number of generators and relatively few relators.
If the '-n' option is called, then this general multiplier is not
computed, and the composite multipliers are built up from the
single basic multiplier automata alone.
\end{description}

The other options are standard, and described in Section 
"Exit Status of Programs and Meanings of Some Options".
\Section{gpmult}

'gpmult [-ip d/s] [-op d/s] [-silent] [-v] [-l] <groupname>'

This calculates the individual multiplier automata, for the monoid
generators of the short-lex automatic group, of which the rewriting-system
defining the group is in the file <groupname>.
It assumes that 'autgroup' (or the three programs 'kbprog\ -wd', 'gpmakefsa'
and 'gpaxioms') have already been run successfully on the group.
Input is from <groupname>.'gm', and output is to <groupname>'.m'<n>
(one file for each multiplier), for $n = 1, \ldots , ng+1$, where $ng$ is
the number of monoid generators of the group. The final multiplier is
the equality multiplier, which accepts $(u,v)$ if and only if $u = v$ and
$u$ is accepted by the word-acceptor.
All of the options are standard.
\Section{gpminkb}

'gpminkb [-op1 d/s] [-op2 d/s] [-silent] [-v] [-l] <groupname>'

Suppose that the file <groupname> contains a rewriting-system
defining a short-lex automatic group,
and that 'autgroup' (or the three programs 'kbprog\ -wd', 'gpmakefsa'
and 'gpaxioms') have already been run successfully on the group.
This program calculates some associated automata, which can be
interesting and useful. Input is from <groupname>'.wa' and from
<groupname>'.diff2'.

Firstly, a finite state automaton which accepts the
minimal reducible words in the monoid generators (i.e. the set of
left-hand-sides  of the (probably infinite) minimal confluent set of
rewrite-rules) for the group is output to <groupname>'.minred'.
Secondly, a two-variable finite state automaton accepting precisely
this minimal confluent set of rewrite-rules
for the group is output to <groupname>'.minkb'.
Finally, the correct minimal first and second word-difference machines are
computed and output to <groupname>'.diff1c' and <groupname>'.diff2c'.
It may be interesting to compare these with <groupname>'.diff1' and
<groupname>'.diff2', but remember that the states may be in a different
order. (The states of a finite state automaton can be re-ordered into a
standard order with the program 'fsabfs'. See Section "fsabfs".)
The file <groupname>'.diff1c' can be used efficiently as input for
'wordreduce'; see Section "wordreduce (automatic groups)" below.

The options are all standard, but note that '-op1' refers to the format
of the one-variable automaton in <groupname>'.minred' (and is dense by
default), whereas '-op2' refers to the two-variable automata in
<groupname>'.minkb', <groupname>'.diff1c' and <groupname>'.diff2c',
and is sparse by default.

\Section{wordreduce (automatic groups)}
'wordreduce  [-kbprog/-diff1/-diff2/-diff1c] [-mrl <maxreducelen>] groupname'

This program can be used either on the output of 'kbprog' or on the output
of the automata package. The '-kbprog' option refers to the former use,
which is described in Section "wordreduce (Knuth--Bendix)".
The '-kbprog' usage is considerably quicker, but it is only guaranteed
accurate when a finite confluent set of equations has been calculated. 
The automatic groups programs are normally only used when such a set
cannot be found, in which case the usage here is the only
accurate one available.
The '-kbprog' usage is the default if the file <groupname>'.kbprog' is present.
To be certain of using the intended algorithm, call one of the flags
'-diff1', '-diff2', '-diff1c' explicitly.
If <groupname>'.kbprog' is not present, then input will be from
<groupname>'.diff2' by default, and otherwise from <groupname>'.diff1'
or <groupname>'.diff1c', according to which option is called.

It is assumed that the file <groupname> contains a rewriting-system
defining a short-lex automatic group,
and that 'autgroup' (or the three programs 'kbprog\ -wd', 'gpmakefsa'
and 'gpaxioms') have already been run successfully on the group.
The '-diff1' option should only be used for input if  'autgroup' or
'gpmakefsa' was run successfully with the '-diff1' option, in which
case it will be the most efficient. (Otherwise you might get wrong answers.)
The  '-diff1c' option can only be used if 'gpminkb' has already been run,
but in that case it will be the most efficient.

'wordreduce' reduces word in the group generators of the group to their
minimal irreducible equivalent under the short-lex ordering.
It can therefore be used to solve the word problem in the group.
'wordreduce' prompts for the words to be input at the terminal.

The option '-mrl\ <maxreducelen>' puts a maximum length on the words to
be reduced. (In the situation here, they can never get longer during reduction.)
The default is 32767.

\Section{fsacount (automatic groups)}
'fsacount  [-ip d/s] [-silent] [-v] [<filename>]'

This is one of the finite state automata functions. See Chapter
"Programs for Manipulating Finite State Automata" for the complete list.
The size of the accepted language is counted, and the answer (which may
of course be infinite) is output to 'stdout'. Input is from <filename> if
the optional argument is present, and otherwise from 'stdin', and it
should be a declaration of a finite state automaton record.

If 'autgroup' (or the three programs 'kbprog\ -wd', 'gpmakefsa'
and 'gpaxioms') have already been run successfully on the group defined
in the file <groupname>, then running 'fsacount <groupname>.wa'
will calculate the size of the group.
For description of options, see Section
"Exit Status of Programs and Meanings of Some Options".

\Section{fsaenumerate (automatic groups)}

'fsaenumerate  [-ip d/s] [-bfs/-dfs] <min> <max> [<filename>]'

This is one of the finite state automata functions. See Chapter
"Programs for Manipulating Finite State Automata" for the complete list.
Input is from <filename> if
the optional argument is present, and otherwise from 'stdin', and it
should be a declaration of a finite state automaton record.
<min> and <max> should be non-negative integers with <min> $\le$ <max>.
The words in the accepted language having lengths at least <min> and at
most <max> are enumerated, and output as a list of words to 'stdout'
or to the file <filename>'.enumerate'.

If 'autgroup' (or the three programs 'kbprog\ -wd', 'gpmakefsa'
and 'gpaxioms') have already been run successfully on the group defined
in the file <groupname>,
then running 'fsaenumerate <min> <max> <groupname>.wa'
will produce a list of elements in the group of which the reduced
word representatives have lengths between <min> and <max>.
If the option '-dfs' (depth-first search - the default) is called,
then the words
in the list will be in lexicographical order, whereas with '-bfs'
(breadth-first-search), they will be in order of increasing length, and in
lexicographical order for each individual length (i.e. in short-lex order).
Depth-first-search is marginally quicker.
For description of other options, see Section
"Exit Status of Programs and Meanings of Some Options".

\Section{gpgeowa}

'gpgeowa [-op1 d/s] [-op2 d/s] [-silent] [-v] [-l/-h] [-f] [-diff1/-diff2]'\\
'| || || || || || || | [-me <maxeqns>] [-mwd <maxwdiffs>] [-n] <groupname>'

This program attempts to construct the geodesic word-acceptor of a
word-hyperbolic group, after the automatic structure has been calculated.
In theory, it works whenever the group is word-hyperbolic, so if it
succeeds, then the group has been proved to have this property (due
to a result proved by Panos Papasoglu in~\cite{PAP}),
although it does not provide an estimate of the hyperbolic constant.
If the group is not word-hyperbolic, then 'gpgeowa' will not terminate.

It is assumed that the file <groupname> contains a rewriting-system
defining a short-lex automatic group,
and that 'autgroup' (or the three programs 'kbprog\ -wd', 'gpmakefsa'
and 'gpaxioms') have already been run successfully on the group.
The '-diff1' option should only be used for input if  'autgroup' or
'gpmakefsa' was run successfully with the '-diff1' option, in which
case it will be the most efficient. (Otherwise you might get wrong answers.)

Input is from <groupname>'.wa' and  <groupname>'.diff2' (or <groupname>'.diff1'
if the '-diff1' option is used). Output (when successful) is to the three
files <groupname>'.geowa', which contains the geodesic word-acceptor,
<groupname>'.geopairs', which contains a two-variable automaton whose
language consists of all pairs of equal geodesic words in the group generators,
and <groupname>.'geodiff', which contains a two-variable word-difference
machine, consisting of the word-differences and transitions necessary for
constructing <groupname>'.geopairs'.

The '-me <maxeqns>', '-mwd <maxwdiffs>' and '-f' options are similar to those in
'gpmakefsa' for example (see Section "gpmakefsa"). Other options are standard.

If the '-n' option is called, then an additional 2-variable automaton
is output to <groupname>'.near\_geopairs'.
This automaton accepts all pairs of geodesic words $u$ and $v$ such
such that $u^{-1}v$ has length at most one as a group element.

\Section{Other programs}

The other programs in the package will be simply listed here. For
further details, look at their source files in the 'src' directory.
If you are sure that you do not want them, and wish to save disk-space,
you can safely delete their executables from the 'bin' directory.
\begin{description}
\item['gpwa'] Calculate word-acceptor.
\item['gpgenmult'] Calculate general multiplier.
\item['gpcheckmult'] Carry out the correctness test on the general multiplier.
\item['gpgenmult2'] Calculate the general multiplier for words of length 2.
\item['gpmult2'] Calculate a particular multiplier for a word of length 2.
\item['gpcomp'] Calculate the composite of two multiplier automata.
\end{description}

\Section{Examples (automatic groups)}

There is a collection of examples of short-lex automatic groups in the
directory 'ag\_data' of widely varying difficulty. The automatic structure
has been calculated and verified as correct in all cases, however.
These can usefully be used as test examples for gaining experience in the
use of the programs. We therefore give a brief summary below of how
they behave.

The 'autgroup' program will complete successfully in its default form on the
following examples, in roughly increasing order of difficulty.
The cpu-times quoted are on a SPARCstation 20.
All 'degen<n>' examples, 'c2', 'ab1', 'f2', 'zw2', 'c5', 'ab2', '333',
'334', 'trefoil', '235', 'gp55', '236', '238', '237', 'f2\_unusual',
'shawcroft', 'torus', '237b' (up to this example, they took less than
10 seconds cpu-time), 'listing', 'f26',
'johnson', '777', 'gunn', 'hamish12', 'G1L1', 's9', 'cox535' (less than
a minute up to this example), 'knot23', 'hamish13', 'hamish23',
'orbifold', 'johnsonc', 'f28', the last taking just under 4 minutes. 
Then, 'edjvet\_94\_1' takes about 18 minutes, 'picard' 35 minutes and 'andre'
90 minutes.  Remember, that the '-f' option can be used for the larger
examples if there are problems with memory or swapping, in which case
the times will be a little longer; for example, 'edjvet\_94\_1' took just under
20 minutes with '-f' on.

For the examples, 'cox33334c' and 'cox5335', the '-diff1' option needs
to be used. The former takes about 17 minutes and the latter 215 minutes.
However, 'cox5335' works much better with the '-l' option on as well, when
it only needs about 63 minutes. The examples 'surgery' and 'johnsonb'
only complete with the '-l' option called, and take 65 and 150 minutes,
respectively. 'f29' does not complete at all by just running 'autgroup -l',
since the limit imposed on the maximum number of equations in 'kbprog' is
still too small. This can be completed with a limit of about 500000 on
the 'maxeqns' parameter, but it produces a temporary file of size about 140
Megabytes during the course of the calculation.

%%File subcos.tex
\Chapter{Subgroup and Coset Files}
\Section{Subgroup files}

In this and the following two chapters, $G$ will denote a group (or
monoid) defined by a rewriting system in the file <groupname>, and $H$ a
subgroup (or submonoid) of $G$.

For any computation involving subgroups (or submonoids) of
a group or monoid, the user must first prepare an input file that
defines generators of the substructure $H$ as words in the generators of
the parent structure $G$.  These substructure generators may also be given
their own names (which we shall refer to as $H$-generators), which must
of course be distinct from those of the generators of $G$ (which we shall
call $G$-generators). If the file containing the rewriting system for $G$ is
called <groupname>, then the file containing the definitions (and possibly
names) of the generators of $H$ should be called <groupname>.<subname>, where
it is recommended that <subname> has the string 'sub' as a prefix. Some
examples can be found in the directory 'subgp\_data'. For example, there is a
file called 'ab2' that defines the 2-generator free abelian group.

|
#free abelian group of rank 2
_RWS := rec(
  isRWS := true,
  ordering := "shortlex",
  generatorOrder := [a,b,A,B],
  inverses := [A,B,a,b],
  equations := [ [b*a,a*b] ]
);
|
and one called 'ab2.sub' that defines a subgroup of index 6.

|
_RWS_Sub := rec(
  subGenerators:=[a^2*b^3, a^4*b^9],
  subGeneratorNames:=[x,y]
);
|

The syntax is self-explanatory. Note that the $G$-generators are <a>, <b>,
<A> and <B>, and the $H$-generators <x> and <y>.
It is important that no generator should be named '\_H' (which is
unlikely anyway), since that will be used by the program 'makecosfile'
(see below) as a symbol that represents the substructure $H$ itself.
The 'subGenerators' list may not be empty or have gaps.
The 'subGeneratorNames' field is optional but, if present, the list defined
must have the same length as that defined by 'subGenerators', and have no gaps.
Another optional field, not present in this example, is
'subGeneratorInverseNames'. This list must have the same length as
'subGeneratorNames', but it may have gaps. The names occurring in it
must all occur in 'subGeneratorNames'. It is used to record the fact that
certain of these generators are inverse to each other; it is the
user\'s responsibility to ensure that this information is accurate.
An example of a file with this feature, specifying another subgroup of the same
group, is as follows.

|
_RWS_Sub := rec(
  subGenerators:=[a^2, A^2, b^5, B^5],
  subGeneratorNames:=[x,X,y,Y],
  subGeneratorInverseNames:=[X,x,Y,y]
);
|

It is not actually necessary to specify inverse generators for subgroups
explicitly in this fashion, since the program 'makecosfile' will adjoin
them automatically if required.

At present, all that can be done with a substructure file is to apply the
program 'makecosfile' to it to produce a coset rewriting system.
Other facilities (such as computing intersections of certain subgroups)
may be provided in the near future. Note, however, that the most important
application is probably for the computation of automatic structures for coset
systems and subgroup presentations. If this is done using the shell script
'autcos', then the user does not need to run 'makecosfile', since it is done
automatically by 'autcos'.

\Section{makecosfile}

'makecosfile [-sg] [-ni] <groupname> [<subname>]'

Since cosets of submonoids of monoids do not in general appear to behave
in a manner suitable for use with with the coset rewriting systems
that are output by 'makecosfile' (for example, if $L$ is a  submonoid of $M$
and $x \in L$, then it does not follow in general that $Lx=L$), we shall
assume that we are dealing with a subgroup $H$ of a group $G$. The program
itself will not attempt to verify this, however, and it is not necessary
for all generators to have inverses defined in the input file.

'makecosfile' takes its input from the file <groupname>, which should contain a
declaration of a record defining a rewriting-system, in the format described
in Chapter "Introduction", and from <groupname>.<subname>, which should
contain a list of subgroup generators and (optionally) names for them,
as described in the preceding section. Output is to the file
<groupname>.<cosname>, and consists of a coset rewriting system,
which will be described in the next paragraph. <cosname> is defined
as follows. If (as is recommended), <subname> has the form 'sub'<string>
for some (possibly empty) string <string>, then <cosname> =
'cos'<string>. Otherwise, <cosname> = <subname>'\_cos'.
Currently, the rewriting system is not allowed to have a weighted-lex ordering.
The default for <subname> is just 'sub', and then
<cosname> = 'cos'.

A coset rewriting system contains the same generators and equations as the
original one defined in <groupname>, and also some additional generators
and equations. In particular, a new symbol named '\_H' (which is not really
a generator of $G$ or of $H$, but is nevertheless defined as one)
will be added that represents $H$ itself.

If the option '-sg' is not called, then this will be the only new generator,
and the new equations will have the form '[\_H\*<w>,\_H]', where <w> is
a word defining one of the subgroup generators, as listed in the
subgroup file. The new symbol '\_H' has no inverse, and so it does not
cancel on the left, and the new equation represents the coset equation
$Hw=H$. The ordering for a coset rewriting system is
always '``wreathprod\'\'', where the level of '\_H' is 1, and those of
the $G$-generators are increased by one if necessary to make them
greater than that of '\_H'. (This is why weighted-lex orderings are not
possible.) In the first example in the preceding section, the
output file 'is ab2.cos'.

|
_RWS := rec(
           isRWS := true,
     isConfluent := false,
         tidyint := 10,
  generatorOrder := [a,b,A,B,_H],
        ordering := "wreathprod",
           level := [2,2,2,2,1],
        inverses := [A,B,a,b,],
       equations := [
         [b*a,a*b],
         [_H*a^2*b^3,_H],
         [_H*a^4*b^9,_H]
       ]
);
|

If the option '-sg' (for ``subgroup generators\'\') is called, then the
record defined in the subgroup file must contain a
'subGeneratorNames' field. The names specified in this list
will also be introduced as new generators in the coset rewriting system,
with level 1. The new equations will now have the form '[\_H\*<w>,<x>\*\_H]',
where <x> is the new name for the subgroup generator <w>. In this
situation, the symbol '\_H' is being used purely as a separator, and the
equation represents an exact equation in the group if '\_H' is ignored.
In general, words in the group can be represented as '<v>\*\_H\*<w>', where
<v> is a word in the $H$-generators, and <w> is a word in the $G$-generators.
This represents the element $vw$ of $G$.

By default, if the subgroup file does not contain a
'subGeneratorInverseNames' field, then inverses of the $H$-generators will be
appended as further $H$-generators. (The inverse symbol for
$H$-generator $x$ is named $x^{-1}$.) If the '-ni' option is called,
however, then these inverse generators are not introduced. By default,
the output of 'makecosfile' applied with the '-sg' option to the example
above is as follows.

|
_RWS_Cos := rec(
           isRWS := true,
     isConfluent := false,
         tidyint := 10,
  generatorOrder := [a,b,A,B,_H,x,y,x^-1,y^-1],
        ordering := "wreathprod",
           level := [2,2,2,2,1,1,1,1,1],
        inverses := [A,B,a,b,,x^-1,y^-1,x,y],
       equations := [
         [b*a,a*b],
         [_H*a^2*b^3,x*_H],
         [_H*a^4*b^9,y*_H]
       ]
);
|

Coset rewriting systems are used as input to the program 'kbprogcos', either
as a standalone (as described in the next chapter), or with the '-wd' option
as part of the automatic coset system package described in Chapter
"The Automatic Coset System and Subgroup Presentation Package".

%%File kbcos.tex
\Chapter{The Knuth--Bendix Program for Cosets of Subgroups}
\Section{kbprogcos}

'kbprogcos  [-f] [-r] [-ro] [-t <tidyint>] [-me <maxeqns>] [-ms <maxstates>]'\\
'| || || || || || || |[-mrl <maxreducelen>] [-mlr <maxlenleft> <maxlenright>]\
[-mo <maxoverlaplen>]'\\
'| || || || || || || |[-sort <maxoplen>] [-v] [-silent]\
[-rk <minlen> <mineqns>]'\\
'| || || || || || || |[-cn <confnum>] <groupname> [cosname]'

'kbprogcos' is a version of 'kbprog' (see Section "kbprog") that is specially
adapted for operating on coset rewriting systems, which were described in
Section "makecosfile".  It can be used either as a standalone or
(with the '-wd' option) as part of the automatic coset system
program 'autcos'. Only its use as a standalone will be described in this
chapter. The other usage will be dealt with in Chapter
"The Automatic Coset System and Subgroup Presentation Package".
All of the options except '-f' are the same as for 'kbprog', and will not
be described here. The '-f' option is described at the end of the section.
Since '``wreathprod\'\''  is the only allowable ordering for words, there are
no ordering options.

'kbprogcos' takes its input from the file <groupname>.<cosname>,
where <cosname> defaults to 'cos'. This should contain a
declaration of a record defining a coset rewriting-system, in the format
described in Chapter "Subgroup and Coset Files".
(Normally  the file <groupname>.<cosname> will have been created
from a rewriting system for a group $G$ in the file <groupname>,
and a subgroup $H$ of $G$, by a call of the program 'makecosfile'.)
Output is to two files, <groupname>.<cosname>'.kbprog' and
<groupname>.<cosname>'.reduce'. The first contains an updated declaration of
the original coset rewriting-system, in which the |equations|\ field contains
the list of all reduction equations found so far. If the process has completed,
and the system is now confluent, then a new field |isConfluent| will
have appeared, and will be set equal to 'true'. In the equations,
the left hand side will always be greater than the right hand side in the
ordering on strings that is being used.
The second file contains a finite state automaton,
which can be used, together with the contents of the first file,
to reduce words and cosets in the group and subgroup generators. This is done
using the program 'wordreduce' with the '-cos' option
(see Section "wordreduce -cos (Knuth--Bendix)").

There are three types of generators in a coset rewriting system,
the $G$-generators, which are the generators of $G$ coming from the
original rewriting system for $G$, the symbol '\_H', which represents the
subgroup $H$ itself and, optionally, the $H$-generators, which should be
distinct from the $G$-generators, and are independent names for
the generators of the subgroup $H$.
There are also three types of equations, those in the $G$-generators alone,
those involving the generator '\_H', and those in the $H$-generators alone.
(Of course, there will be none of the third type if $H$-generators are
not present.) The left and right hand sides of the equations involving
'\_H' should have the form <v>'\_H'<w>, where <v> is a (possibly empty)
word in the $H$-generators, and <w> a (possibly empty) word in the
$G$-generators. In fact, those in the input file should all have the
form '[\_H'<w>,'\_H]', or '[\_H'<w>,<x>'\_H]' for some $H$-generator <x>,
and they define the generators of $H$ as words <w> in the $G$-generators.
The order of the equations in the input file is unimportant, but in the
output file, the equations contaiaing '\_H' will come first, then the
$G$-equations, and finally the $H$-equations (if any).
The ordering of words in a coset rewriting system must be '``wreathprod\'\'',
and '\_H' and the $H$-generators must all have a strictly smaller level than
each of the $G$-generators. This is to ensure that words containing
'\_H' get themselves reduced to the desired form <v>'\_H'<w> with the
word <w> coming as early as possible in the ordering of the $G$-words.

The previous paragraph can be taken as a definition of a coset rewriting
system. Any input not conforming to these rules will either be rejected
as invalid by 'kbprogcos', or will produce meaningless results.
Of course, if the input is produced using 'makecosfile' as recommended,
then it will be guaranteed to be valid.

If the output system is confluent, then the sets of the three types of
equations will be confluent individually in the following sense.
The $G$-equations will form a confluent system for $G$, just as they
would if 'kbprog' had been run on $G$. 
The equations involving '\_H' together with the $G$-equations will
be sufficient to reduce any coset '\_H'<w>, with <w> a word in the
$G$-generators, to '\_H'$w^\prime$ (or <x>'\_H'$w^\prime$ if $H$-generators are
present), where $w^\prime$ is the least word (under the ordering being used)
in the $G$-generators that lies in the coset $Hw$. If $H$-generators are
present, then the equality $w = xw^\prime$ will hold in $G$.
The $H$-equations, if present, will form a confluent rewriting system for
$H$ (and so we will, in particular, have computed a presentation for the
subgroup $H$).

Of course, for infinite groups, 'kbprogcos', like 'kbprog',
will more often than not run forever by default.
The halting conditions that can be imposed by the user, either by
setting appropriate fields in the input file, or by using
command-line options, are the same as for 'kbprog', so we will not
repeat their description here.  As with 'kbprog',
it is also possible to halt the program interactively at any stage, by
sending it a single interrupt signal (which usually means typing Control-C).
If you do this, then it will halt and output at the next convenient
opportunity, and so you may have to wait a while. If you send two interrupt
signals, then it will abort immediately without outputting.

*The '-f' option*\\
This option is only useful when the index of $H$ in $G$ is finite.
The theory behind it is described in Section 3.10 of \cite{Sims94}.

When '-f' is called, if 'kbprogcos' finds at some stage that there are only
finitely many irreducible words of the form '\_H'<w> (which implies
that $\|G\:H\|$ is finite), then it halts after making a limited number of
further overlap tests, without the system necessarily being confluent.
At this stage, it is guaranteed that all words of the form '\_H'<w>
reduce under the system to '\_H'$w^\prime$ (or <x>'\_H'$w^\prime$), where
$w^\prime$ is the correct minimal word that lies in in the coset '\_H'<w>.
In other words, it reduces cosets of $H$ correctly, even though it may not
reduce all $G$-words or $H$-words correctly. However, if $H$-generators are
present, then it is also guaranteed that the $H$-equations will form a
presentation of $H$ (although not necessarily a confluent one).
The index of $H$ in $G$ will be equal to the number of irreducible words of
form '\_H'<w>, and will be printed as a message by 'kbprogcos' before it
halts.

Conversely, if the index of $H$ in $G$ is finite, and the '-f' option is
called, then it can be proved that the algorithm employed will eventually
halt, although it may take a long time. So it can be used as an
alternative to Todd-Coxeter coset enumeration for calculating $\|G\:H\|$, and
to the modified Todd-Coxeter method of calculating subgroup presentations.
However, it does appear that the Todd-Coxeter based methods are more
efficient in most examples.

*Exit status*\\
The exit status is 0 if either 'kbprogcos' completes with a confluent set of
equations, or if the '-f' option is called and it halts with correct
reduction of cosets. The exit status is
2 if 'kbprogcos' halts and outputs a non-confluent set because some limit has
been exceeded, or it has received an interrupt signal, and 1 if it exits
without output, with an error message.

\Section{wordreduce -cos (Knuth--Bendix)}
'wordreduce -cos [-kbprog/-diff1/-diff2] [-mrl <maxreducelen>]'\\ 
'| || || || || || || || || || || |<groupname> [<cosname>]'

'wordreduce -cos' can be used either on the output of 'kbprogcos' or on the
output of the automatic coset system package. The '[-diff<x>]' options refer to
the latter use, which is described in Section
"wordreduce -cos (automatic coset systems)".
The former use is the default if the file <groupname>.<cosname>'.kbprog' is
present; to be certain, call the '-kbprog' option explicitly.
As with 'kbprogcos', <cosname> defaults to 'cos'.

'wordreduce -cos' reduces words using the output of a run of 'kbprogcos',
which is read from the files <groupname>.<cosname>'.kbprog' and
<groupname>.<cosname>'.reduce'.  These words should either (in the notation of
the preceding section) be in the $G$-generators alone, the $H$-generators
alone, or of the form <v>'\_H'<w>, where <v> is a (possibly empty)
word in the $H$-generators, and <w> a (possibly empty) word in the
$G$-generators. If there are no $H$-generators, then the words '\_H'<w>
represent cosets of $H$ in $G$. Otherwise they represent the element
<vw> of $G$, and the symbol '\_H' is being used as a separator.

The reductions will always be correct in the sense that the output word will
represent the same group element or coset as the input word. If the system of
equations in <groupname>.<cosname>'.kbprogcos'
is confluent, then the reduction will be to the minimal word that represents
the group element or coset under the ordering on strings of generators that
was used by 'kbprogcos'. Note that the ordering used for coset rewriting
systems ensures that in a mixed word of form <v>'\_H'<w>, the reduction
will be to $v^\prime$'\_H'$w^\prime$ for smallest possible $w^\prime$, and
so $w^\prime$ is the minimal coset representative of '\_H'<w>.
It can therefore be used to solve the word problem in the group $G$ and
the membership problem for the subgroup $H$.

If the program halted normally on using the '-f' option of 'kbprogcos',
then the the reduction will be correct on cosets of $H$, even though the
system may not be confluent, and not all $G$-words (or $H$-words) will
reduce correctly.

The option '-mrl\ <maxreducelen>' is the same as in 'kbprogcos'.

\Section{fsacount and fsaenumerate(Knuth--Bendix)}

If these programs are run on the output automata in the file
<groupname>.<cosname>'.reduce', then they will count or enumerate all
accepted words of the automata, which include the $G$-words, the
words '\_H'<w> (or <x>'\_H'<w>) and (if present) the $H$-words.
So, if $H$ is a subgroup of order $l$ and index $m$ in a group
of order $n$ then, if $H$-generators are not present, there will be
$n+m$ accepted words (for the group elements and the cosets of $H$).
If $H$-generators are present, then the number of accepted words will
be $2n+l-1$, comprising $n$ $G$-words, $n$ mixed words of form <x>'\_H'<w>
(one for each group element), and $l$ $H$-words. However, the empty word
is both a $G$-word and an $H$-word, which accounts for the $-1$ in the
expression. 

It is possible to count or enumerate just the coset representatives
$w$ occurring in the cosets '\_H'<w> or mixed words  <x>'\_H'<w>, by
calling 'fsacount' or 'fsaenumerate' with the option '-is 2'.
This re-defines the initial state of the automaton to be 2, which has
the effect that the words counted or enumerated are those of the
form '\_H'<w>, but with the '\_H' removed. If $H$ has finite index
$m$ in $G$, then there will be exactly $m$ such words $w$.

\Section{Examples of Knuth--Bendix on coset rewriting systems}

The 'subgp\_data' directory contains a number of examples of rewriting
systems for groups together with some defined subgroups. Some of these,
including the Coxeter groups, 'gunn' and 'f28', are intended for use with
the automatic coset structure programs described in Chapter
"The Automatic Coset System and Subgroup Presentation Package",
and 'kbprogcos' will not complete on them.

The example 'ab2' is the free abelian group of rank 2, and the subgroup
defined in 'ab2.sub' has index 6. First run

|
 >makecosfile ab2
|

to create a coset rewriting system for this subgroup (without $H$-generators)
in the file 'ab2.cos'. If 'kbprogcos' is run with no options on this system,
then it does not halt, but with the '-f' option we get\:

|
 >kbprogcos  -f ab2
#Coset language has size 6. Setting maxoverlap length to 7.
#System is not guaranteed confluent, since not all overlaps were processed.
#Output\:\ 28 eqns; table has 24 states.
#Halting with 28 equations.
|

and from the first line of the output we know that the index is 6.
We can check this by using 'fsacount'. To count the cosets alone
(omitting the $G$-words, which correspond to  elements of $G$), use
the option '-is 2', as described above.

|
 >fsacount ab2.cos.reduce
#The language accepted is infinite.
 >fsacount -is 2 ab2.cos.reduce
#The language accepted has size 6.
 >
|

The example '8l32' is a finite group of order 1344, and '8l32.sub' is a
normal subgroup of order 8 and index 168. (The quotient group is
${\rm PSL}(3,2)$.) Below, we introduce $H$-generators, by using the '-sg'
option with 'makecosfile'.

|
 >makecosfile -sg 8l32
 >kbprogcos 8l32
#System is confluent.
#Output\:\ 828 eqns; table has 724 states.
#Halting with 828 equations.
 >fsacount 8l32.cos.reduce
#The language accepted has size 2695.
 >fsacount -is 2 8l32.cos.reduce
#The language accepted has size 168.
|

The 2695 words accepted by the word-reduction automaton consist of 1344
$G$-words, 1344 mixed words, and 8 $H$-words, where the empty word is both
a $G$-word and an $H$-word. When the initial state is set to 2, we just get
the coset representatives. They can be enumerated in length-increasing
order by running

|
 >fsaenumerate -is 2 -bfs 0 200 8l32.cos.reduce
|

and the list of words is output to the file '8l32.cos.reduce.enumerate'.
We used a large upper length limit (200) for the search to ensure that we got
everything.

% rewriting equations of the form
%File autcos.tex
\Chapter{The Automatic Coset System and Subgroup Presentation Package}

The aim of this package is firstly to calculate the finite state automata
associated with an automatic coset system for a suitable subgroup $H$ of a
short-lex automatic group $G$, and secondly to calculate a
presentation of $H$. At the end of a successful calculation, five automata
will be stored. These are the first and second word-difference machines, the
coset word-acceptor, and two versions of the general-multiplier for the coset
structure. The word-difference machines and one version of the multiplier
are midfa (multiple-initial-state deterministic automata). This means that
their transition tables are deterministic, but they may have more than one
initial states. The coset word-acceptor (which accepts the short-lex least
word that lies in each right coset of $H$ in $G$) and the other version of the
multiplier are standard deterministic automata.  The multiplier is a
two-variable automaton over the set of monoid generators of $G$ that accepts a
pair of words $(w_1,w_2)$ in these generators if and only if $w_1$ and $w_2$
are accepted by the coset word-acceptor, and the right cosets $Hw_1x$ and
$Hw_2$ are equal for some monoid generator $x$ of $G$. As is the case with the
general multiplier for an ordinary automatic structure, the success states
of the multiplier are labeled by the generators $x$.
Currently, the only reference for the theory of automatic coset systems is the
Ph.D. thesis of Ian Redfern \cite{Red93}. It is proved there that quasiconvex
subgroups of word-hyperbolic groups always have short-lex automatic coset
systems.

The resulting automata can be used for reducing words $w$ to the unique
minimal word $v$ in the group under the short-lex ordering, such that $Hv=Hw$.
In particular, this enables the generalised word-problem to be solved for the
subgroup $H$ of $G$. They can also be used for counting and enumerating the
accepted language.
See Sections "wordreduce -cos (automatic coset systems)", and
"fsacount and fsaenumerate (automatic coset systems)" for details.
(In fact the multiplier automaton is not needed for any of these processes,
but it forms part of the automatic coset system from a
theoretical viewpoint, so we regard it as part of the output of the
calculation.)

There are two possibilities for computing the automatic coset system.
The simplest is to use the program 'autcos', which is in fact a
Unix Bourne-shell script, which runs the C-programs involved,
and attempts to make a sensible choice of options.
The other possibility is to run the programs 'makecosfile' (with the '-sg'
option), 'kbprogcos\ -wd', 'gpmakefsa\ -cos', 'gpaxioms\ -cos' and possibly
'gpsubpres' individually, and to select the options oneself.
'kbprogcos -wd' runs the Knuth--Bendix process on the coset rewriting system
for the subgroup $H$ of $G$, and calculates the resulting word-differences
arising from the equations generated.

For the complete process to be successful, the
the set of all word-differences arising from these equations,
and the set of all words $v$ in $H$ that occur in the rewriting equations of
the form '[\_H'$w_1,v$'\_H'$w_2$']' must both be finite;
however, 'kbprogcos -wd' itself will normally run forever, generating
infinitely many equations, and so sensible halting criteria must be devised.
Clearly one wants to halt as soon as all of the
word-differences and words $v$ in $H$ have been found, if possible. Further
details are given in Section "kbprogcos -wd". 'gpmakefsa\ -cos' uses the
word-difference output by 'kbprogcos\ -wd' to compute the word-acceptor and
multiplier automata.
It performs some checks on these which can reveal if the set of
word-differences (or subgroup words) used was not in fact complete, and find
some of the missing ones. It can then try again with the extended sets of
word-differences. After this process completes successfully, the program
'gpaxioms\ -cos' performs the axiom-checking process, which proves the
correctness of the automata calculated.

If a presentation of the subgroup $H$ is required, then it can be computed
using the program 'gpsubpres' (see "gpsubpres"). Currently this will be on
a set of generators for $H$ determined by the program. (This is in fact
corresponds to the final set of initial states that occur in the midfa version
of the generalised multiplier.) It is hoped that it
will eventually be possible to compute such a presentation on the set
generators of $H$ originally specified by the user in the file defining $H$.
The presentation will be highly redundant, with multiple occurrences of each
relator and its cyclic conjugates. It is therefore output as a {\GAP}
definition of a finitely presented group, so that it can read in by {\GAP}
and simplified if required.

If the file containing the coset rewriting system is named
<groupname>'.'<cosame>, then all files created by the programs will have names
of form <groupname>'.'<cosname>'.'<suffix>, except for that containing the
presentation of $H$, which will be named <groupname>'.'<subname>'.pres'.

\Section{autcos}

'autcos  [-l] [-v] [-silent] [-diff1] [-f] [-p] [-pref <prefix>] <groupname>'\\
'| || || || || || || |[<subname>]'

Compute the finite state automata that constitute an automatic coset system,
for a suitable subgroup $H$ of a short-lex automatic group $G$.
Input is taken from the files <groupname> and <groupname>'.'<subname>.
As with 'makecosfile', the default for <subname> is 'sub'.
The first of these files should contain a declaration of a record defining a
rewriting-system for $G$, in the format described in Chapter "Introduction".
Since $G$ is a group $G$, all monoid generators must have
inverses listed explicitly. The second input file should contain a record
defining a subgroup $H$ of $G$ in the format described in Chapter
"Subgroup and Coset Files", and the 'subGeneratorNames' field must be present.
Output is to the files <groupname>'.'<cosname>'.midiff1',
<groupname>'.'<cosname>'.midiff2' (midfa word-difference machines),
<groupname>'.'<cosname>'.wa' (coset word-acceptor),
<groupname>'.'<cos-name>'.migm' and <groupname>'.'<cosname>'.gm'
(midfa genemral multiplier and deterministic general multiplier), and
possibly <groupname>'.'<subname>'.pres' (subgroup presentation), where
the string <cosname> is derived from <subname> by substituting
'cos' for 'sub' as described in Chapter "Subgroup and Coset Files".

*Options*\\
For a general description of the options '[-l]', '[-v]' and '[-silent]',
see Section
"Exit Status of Programs and Meanings of Some Options".
For greater flexibility in choice of options, run the programs
'makesubfile\ -sg', 'kbprogcos\ -wd', 'gpmakefsa\ -cos', 'gpaxioms\ -cos' and
'gpsubpres' individually, rather than using 'autcos'. 

\begin{description}
\item[|-l |]
This causes 'gpmakefsa\ -cos', 'gpaxioms\ -cos' and 'gpsubpres' to be run with
the '-l' option, which means large hash-tables.
However, '-l' also causes 'kbprogcos\ -wd' to be run
with larger parameters, and you are advised to use it only after you
have tried first without it, since it will cause easy examples to take
much longer to run.
\item[|-diff1|]
This causes 'gpmakefsa\ -cos' to be run with the '-diff1' option. See Section
"gpmakefsa -cos" for further details.
\item[|-f |] This causes 'gpmakefsa\ -cos' and 'gpaxioms\ -cos' to be run with
the '-f' option. See Sections "gpmakefsa -cos" and "gpaxioms -cos" for
further details. This is the option to choose if you need to save space,
and don\'t mind waiting a bit longer.
\item[|-p|] Calculate a presentation of the subgroup $H$. This is likely to be
the most time- and space-consuming part of the whole computation, and is not
done by default.
\item[|-pref| <prefix>] Name the finite presentation of the subgroup
$H$ that is output in {\GAP} format <prefix>, and its generators
<prefix>'1', <prefix>'2', etc. The default for <prefix> is '\_x'. 
\end{description}
\Section{kbprogcos -wd}
'kbprogcos -wd [-t <tidyint>] [-me <maxeqns>] [-ms <maxstates>]\
[-mwd <maxwdiffs>]'\\
'| || || || || || || |[-hf <halting\_factor>] [-mt <min\_time>]\
[-rk <minlen> <mineqns>]'\\
'| || || || || || || |[-v] [-silent] [-cn <confnum>] <groupname> [<cosname>]'

Some of these options are of course the same as for when 'kbprogcos' is being
used as a stand-alone Knuth--Bendix program for cosets of subgroups --
see Section "kbprogcos". The default value of <cosname> is 'cos'.

'kbprogcos\ -wd' takes its input from the file <groupname>.<cosname>,
which should contain a declaration of a record defining a coset
rewriting-system for a subgroup $H$ of a group $G$, in the format described in
Chapter "Subgroup and Coset Files". See also Section "kbprogcos", for a
definition of a coset rewriting system. To obtain meaningful results,
$H$-generators must be present in the rewriting system.
The ordering must be |"wreathprod"|, and all $H$-generators  and all
$G$-generators must have the same levels $l_H$ and $l_G$, with $l_H \< l_G$.
Since $G$ and $H$ must both be groups, all $G$-generators and all
$H$-generators must have inverses, and the inverse function must be involutory.
The separator symbol '\_H' should not, of course, have an inverse.
This input file will normally be generated by running 'makecosfile' with the
'-sg' option.

The Knuth--Bendix completion process is carried out on the coset rewriting
system.
However, unlike in the standalone usage of 'kbprogcos', the extended set of
equations and the reduction automaton are not output. Instead, the two
word-difference automata arising from the equations are output when the program
halts. These are printed to the two files <groupname>'.midiff1' and
<groupname>'.midiff2'. The difference between them is that the first automaton
contains only those word-differences and transitions that arise from the
equations themselves. For the second automaton, the set of word-differences
arising from the equations is closed under inversion, and all possible
transitions are included. So the second automaton has both more states and
generally more transitions per state than the first. For their uses, see
Section "gpmakefsa -cos" below.

These word-difference automata are midfa; that is they have deterministic
transition tables, but multiple initial states. They are two-variable automata
in the $G$-generators only. The equations in the $G$-generators are
handled as they are in 'kbprog -wd'. The coset equations
'[\_H'$w_1,v$'\_H'$w_2$']', where $w_1$ and $w_2$ are words in the
$G$-generators and $v$ is a word in the $H$-generators, are handled slightly
differently. The word $v$ is rewritten as a word in the $G$-generators, and
then becomes an initial state of the word-difference machines, which leads
to success on reading the pair of $G$-words '['$w_1,w_2$']'. The equations
in the $H$-generators only are not used. 

As with 'kbprog -wd', this program will not usually complete with a
confluent set of equations, but will continue indefinitely, and so halting
criteria have to be chosen. This can only be usefully done if the sets of
word-differences and initial states arising from the equations are both
finite. This is known theoretically to be the case for quasiconvex subgroups
of word-hyperbolic groups, but fortunately it appears to be true also
for many other subgroups of short-lex automatic groups.
The general idea is to wait until these sets appears to have become constant,
and then stop.

The methods available for interrupting 'kbprogcos -wd', or for specifying
halting criteria with command-line options, are essentially the same as for
'kbprog -wd', and the reader should refer to Section "kbprog -wd" for guidance.
The other options are also the same as there.

*Exit status*\\
The exit status is 0 if 'kbprog' completes with a confluent set of equations,
or if it halts because the condition in the '-hf <halting\_factor>' option
has been fulfilled,
2 if it halts and outputs because some other limit has
been exceeded, or it has received an interrupt signal, and 1 if it exits
without output, with an error message.
This information is used by the shell-script 'autcos'.

\Section{gpmakefsa -cos}

'gpmakefsa  -cos [-diff1/-diff2] [-me <maxeqns>] [-mwd <maxwdiffs>] [-l]'\\
'| || || || || || || || || || |[-ip d/s[dr]] [-opwa d/s] [-opgm d/s]\
[-v] [-silent]'\\
'| || || || || || || || || || |<groupname> [<cosname>]'

Construct the coset word-acceptor and general multiplier finite state automata
for the coset automatic system, of which the rewriting-system defining
the group and subgroup is in the file <groupname>.<cosname>.
The default value of <cosname> is 'cos'.
It assumes that 'kbprogcos\ -wd' has already been run on the system.
Input is from <groupname>.<cosname>'.midiff2' and, if
the '-diff1' option is called, also from <groupname>.<cosname>'.midiff1'. The
coset word-acceptor is output to <groupname>.<cosname>'.wa' and the general
coset multiplier to <groupname>.<cosname>'.migm' and
<groupname>.<cosname>'.gm' in its midfa and deterministic versions,
respectively.  Certain correctness tests are carried out on the
constructed automata. If they fail, then new word-differences will be
found and <groupname>.<cosname>'.midiff2' (and possibly also
<groupname>.<cosname>'.midiff1')
will be updated, and the multiplier (and possibly the word-acceptor)
will then be re-calculated.

There are programs available which construct these automata individually,
and also one which performs the correctness test, but it is unlikely that
you will want to use them. They are mentioned briefly in
Section "Other coset automata programs".

The choice of input, as controlled by the '-diff1' and '-diff2'
options ('-diff2' is the default) is essentially the same as
for 'gpmakefsa' without the '-cos' option; see Section "gpmakefsa".
The correctness check on the multiplier is also similar. The
midfa version of the multiplier is constructed first, and the
deterministic version is constructed from this, and used for the
correctness test. Thus both versions need to be constructed for
each call of the correctness test.
All other options are as for 'gpmakefsa', but note that the
'-f' option is not available in the cosets version.

\Section{gpaxioms -cos}

'gpaxioms [-cos] [-ip d/s[dr]] [-op d/s] [-silent] [-v] [-l] [-n]'\\
'| || || || || || || || || || |<groupname> [<cosname>]'

This program performs the axiom-checking process on the coset word-acceptor and
general multiplier of the automatic coset system, of which the
rewriting-system defining the group and subgroup is in the file
<groupname>.<cosname>. The default value of <cosname> is 'cos'.
It assumes that 'kbprogcos\ -wd' and 'gpmakefsa\ -cos' have already been run
on the system.
It takes its input from <groupname> and <groupname>.<cosname>'.gm'. 
There is no output to files (although plenty of intermediate files are
created and later removed during the course of the calculation).

If successful, there will be no output apart from routine reports, and the exit
status will be 0. If unsuccessful, a message will be printed to 'stdout'
reporting that a relation test failed for one of the group relations, and
the exit status will be 2.

The |-n| option is the same as in Section "gpaxioms".
The other options are all standard, and described in Section 
"Exit Status of Programs and Meanings of Some Options".
\Section{gpmimult}

'gpmimult [-ip d/s] [-op d/s] [-silent] [-v] [-l] [-pref <prefix>]'\\
'| || || || || || || || || || |<groupname> [<cosname>]'

This calculates the individual midfa coset multiplier automata, for the monoid
generators of the group, of which the rewriting-system
defining the group and subgroup is in the file <groupname>.<cosname>.
The default value of <cosname> is 'cos'.
It assumes that 'autcos' (or the programs 'kbprogcos\ -wd', 'gpmakefsa\ -cos'
and 'gpaxioms\ -cos') have already been run successfully on the system.
Input is from <groupname>.<cosname>.'migm', and output is to
<groupname>.<cosname>'.mim'<n>
(one file for each multiplier), for $n = 1, \ldots , ng+1$, where $ng$ is
the number of monoid generators of the group. The final multiplier is
the coset equality multiplier, which accepts $(u,v)$ if and only if
$u = v$ and $u$ is accepted by the coset word-acceptor.

The labels for the initial states of these multipliers correspond
precisely to the generators of $H$ on which a presentation of $H$
can be calculated by the program 'gpsubpres' (see below). They are
described as words in the $G$-generators in the file
<groupname>.<cosname>.'migm', but in the individual multiplier
files, they are given new and distinct names of the form
<prefix>'1', <prefix>'2', etc.
The string <prefix> can be set with the '-pref' option, the
default being '\_x'. All other options are standard.

\Section{gpsubpres}
'gpsubpres [-ip d/s] [-op d/s] [-silent] [-v] [-l] [-pref <prefix>]'\\
'| || || || || || || || || || |<groupname> [<subname>]'

Calculate a presentation of the subgroup $H$ of the group $G$ as defined
by the rewriting system for $G$ in the file <groupname> and the
definition of the subgroup $H$ in the file <groupname>.<subname>.
Here <cosname> is derived from <subname> by substituting
'cos' for 'sub' as described in Chapter "Subgroup and Coset Files".
It assumes that 'autcos' (or the programs 'makesubfile\ -sg', 'kbprogcos\ -wd',
'gpmakefsa\ -cos' and 'gpaxioms\ -cos') have already been run successfully on
$G$ and $H$.
Input is from <groupname> and <groupname>.<cosname>'.migm'
(where <cosname> is derived from <subname> as described in
Chapter "Subgroup and Coset Files" or Section "autcos"), and output
is to the file <groupname>.<subname>'.pres'.

The output presentation, which will usually contain many superfluous and
repeated relators, is a {\GAP} definition of a finitely presented group, which
can be read in by {\GAP} and simplified if required. The name of the group in
this definition is <prefix>, and its generators are named <prefix>'1',
<prefix>'2', etc.  The string <prefix> is equal to '\_x' by default, but can be
set by using the '-pref' option.  The other options are the same as for
'gpaxioms'.

\Section{wordreduce -cos (automatic coset systems)}
'wordreduce [-cos] [-kbprog/-diff1/-diff2] [-mrl <maxreducelen>]'\\
'| || || || || || || || || || || |<groupname> [<cosname>]'

This program can be used either on the output of 'kbprogcos' or on the output
of the automatic coset system package. The '-kbprog' option refers to the former
use, which is described in Section "wordreduce -cos (Knuth--Bendix)".
The '-kbprog' usage is considerably quicker, but it is only guaranteed
accurate when a finite confluent set of equations has been calculated. 
The automatic coset system programs are normally only used when such a set
cannot be found, in which case the usage here is the only
accurate one available.
The '-kbprog' usage is the default if the file <groupname>'.kbprog' is present.
To be certain of using the intended algorithm, call one of the flags
'-diff1', '-diff2' explicitly.
If <groupname>.<cosname>'.kbprog' is not present, then input will be from
<groupname>.<cosname>'.midiff2' by default, and otherwise from
<groupname>.<cosname>'.midiff1'.

It is assumed that the file <groupname>.<cosname> contains a coset
rewriting-system for a subgroup $H$ of a short-lex automatic group $G$,
and that 'autcos' (or the programs 'kbprogcos\ -wd', 'gpmakefsa\ -cos'
and 'gpaxioms\ -cos') have already been run successfully on the system.
The '-diff1' option should only be used for input if  'autcos' or
'gpmakefsa\ -cos' was run successfully with the '-diff1' option, in which
case it will be the most efficient. (Otherwise you might get wrong answers.)

The words to be reduced by 'wordreduce\ -cos' must either be in the
$G$-generators of the coset system alone, in which case the reduction
will be to the short-lex minimal representative of the word in $G$, or
they must have the form $w_1$'\_H'$w_2$, where $w_1$ and $w_2$ are both words
in the $G$-generators, in which case the reduction will be to a word
$w_1^\prime$'\_H'$w_2^\prime$, where $w_1w_2$ and $w_1^\prime w_2^\prime$ are
equal elements in $G$, and $w_2^\prime$ is the short-lex least word that
represents an element of the coset $Hw_2$.  It can therefore be used to solve
the membership problem for the subgroup $H$ of $G$.
Note that the usage is not quite the same as in Section
"wordreduce -cos (Knuth--Bendix)", since the names for the $H$-generators
cannot be used.

The option '-mrl\ <maxreducelen>' puts a maximum length on the words to
be reduced. The default is 32767.

\Section{fsacount and fsaenumerate (automatic coset systems)}

If 'autcos' (or the programs 'kbprogcos\ -wd', 'gpmakefsa\ -cos'
and 'gpaxioms\ -cos') have already been run successfully on the 
coset rewriting system for the subgroup $H$ of the group $G$
in the file <groupname>.<cosname>, then running
'fsacount <groupname>.<cosname>.wa'
or 'fsaenumerate <min> <max> <groupname>.<cosname>.wa'
will calculate the index of $H$ in $G$, or enumerate words that are the
short-lex least representatives of the cosets of $H$ in $G$, and
have lengths between <min> and <max>.
For description of options, see Section
"Exit Status of Programs and Meanings of Some Options".

\Section{Other coset automata programs}

The other programs in the package will be simply listed here. For
further details, look at their source files in the 'src' directory.
\begin{description}
\item['gpwa\ -cos'] Calculate coset word-acceptor.
\item['gpmigenmult'] Calculate midfa general coset multiplier.
\item['midfadeterminize'] To calculate the deterministic general coset
multiplier.
\item['gpcheckmult\ -cos'] Carry out the correctness test on the
general coset multiplier.
\item['gpmigenmult2'] Calculate the midfa general coset multiplier for words
of length 2.
\item['gpmimult2'] Calculate a particular midfa coset multiplier for a word of
length 2.
\item['gpmicomp'] Calculate the midfa composite of two midfa multiplier
coset automata.
\end{description}

\Section{Examples (automatic coset systems)}

There is a collection of examples of short-lex automatic groups  and subgroups
on which 'autcos' will complete in the directory 'subgp\_data'.

%%File subwa.tex
\Chapter{Creating Subgroup Word Acceptors}
\Section{Subgroup word acceptors}
The programs described in this chapter are for creating word-acceptors for
suitable subgroups $H$ of short-lex automatic groups $G$. Such a
word-acceptor is a finite state automaton with alphabet equal to the
monoid generating set for $G$, which accepts precisely those words that lie
in $H$ and are accepted by the word-acceptor for the short-lex automatic
structure of $G$. Thus they accept a unique word for each element of $H$.
An arbitrary word in the $G$-generators can then be tested for membership in $H$
by first reducing to its $G$-normal form (using 'wordreduce') and then
testing for acceptance by the word-acceptor for $H$. Thus the membership
problem (that is the generalized word problem) can be solved for $H$.

Theoretically, if $G$ is short-lex automatic, and its automatic structure
has been computed using 'autgroup' (or its constituents), then this method
will work precisely when $H$ is a quasiconvex subgroup of $G$. The algorithm
used consists of two parts, the construction of a candidate $W_H$ for the
word-acceptor, and a correctness verification. In case of failure of the
second part, one or more $G$-reduced words are calculated that lie in $H$
but are not accepted by $W_H$, and these can be used when the
construction part is repeated. Thus the two parts of the procedure are run
alternately until the correctness test succeeds. The method is based
roughly on that described by Ilya Karpovich in \cite{Kar94}. The correctness
testing part is identical to that described there, but the construction part
uses Knuth-Bendix based methods, rather than Todd-Coxeter coset enumeration
(which was suggested by Karpovich), since this is more in keeping with the
{\KBMAG} package as a whole.

The user can either run the program 'gpsubwa' (described in the following
section), which is a shell script that calls the two component programs
repeatedly until the correctness test is passed, or alternatively these
two components 'gpmakesubwa' and 'gpchecksubwa' (described in the remaining
sections of the chapter) can be run directly by the user. This allows a
slightly more flexible choice of options, but does not appear to be really
necessary in most examples. In fact, in the examples supplied in the
{\KBMAG} 'subgp\_data' directory, only 'l28.sub' fails to pass the
correctness test first time.

In either case, before running either 'gpsubwa' or 'gpmakesubwa',
the user must first run 'autgroup' or its constituents
(see Chapter "The Automatic Groups Package") on the group $G$, and must
provide a means for performing coset reduction of words <w> in the
$G$-generators to their short-lex least representative word <v> with
$Hw = Hv$. There are two possible ways of achieving this. The first is
to run 'autcos' (or its constituents) on the subgroup $H$ of $G$, as
described in Chapter
"The Automatic Coset System and Subgroup Presentation Package".
If successful, this assures correct coset reduction for all words <w>.
However, this may be too time-consuming, or even impossible, in
some examples. In general, it is sufficient to run 'kbprogcos' (see
Section "kbprogcos") on the cosets of $H$ in $G$ for a sufficiently
long time to enable the coset reduction of the required words to
be computed accurately using the reduction automaton that is output
by 'kbprogcos'. Of course the user has no theoretical means of
knowing how long this will be, and 'kbprogcos' will not normally
complete naturally. In practice this does not normally seem to be a problem,
however, and if 'gpsubwa' completes successfully, then the results are
theoretically guaranteed to be correct, irrespective of which
coset reduction method was used. Of course, if 'kbprogcos' had not
been run for long enough, then 'gpsubwa' would never complete, or would exit
abnormally with an error message containing the phrase
|error tracing word|.  If, in a particular example, 'gpsubwa' appears to be
looping indefinitely, or it exits with such a message, then the user should try
re-running 'kbprogcos' for longer than before.


\Section{gpsubwa}

'gpsubwa [-diff1/-diff1c] [-diff1cos/-diff2cos] [-l] [-v] [-silent] [-f]'\\
'| || || || || || || || |<groupname> [<subname>]'

'gpsubwa' attempts to compute a subgroup word-acceptor for the subgroup $H$ of
the short-lex automatic group $G$, where the rewriting system for $G$ is
defined in <groupname>, and the generators for $H$ are defined in the file,
<groupname>.<subname>, where <subname> defaults to 'sub'.
(See Chapter "Subgroup and Coset Files" for information on the format of
the latter file.) The computed automaton is output to the file
<groupname>.<subname>'.wa'.

'gpsubwa' requires the word-acceptor and general multiplier for the short-lex
automatic structure of $G$, which it inputs from the files <groupname>'.wa'
and <groupname>'.gm'.  These need to be computed
first by running 'autgroup' or its constituents. It also requires
word-reduction automata for $G$ and for cosets of $H$ in $G$, as described
above. These are input, respectively,  from <groupname>'.diff2' and
<groupname>.<cosname>'.kbprog' by default, but these defaults may be changed
by using the appropriate options (see below). Here <cosname> is derived
from <subname> by substituting the string 'cos' for 'sub' as described in
Chapter "Subgroup and Coset Files".

'gpsubwa' is a Bourne-shell script that runs the two programs 'gpmakesubwa'
and 'gpchecksubwa', documented in the following two sections, alternately.

*Options*\\
\begin{description}
\item[|-diff1/-diff1c|] Take the input automaton for word-reduction in $G$
from <groupname>'.diff1' or <groupname>'.diff1c', rather than from
<groupname.>'diff2'.
\item[|-diff1cos/-diff2cos|] Take the input automaton for coset word-reduction
of words in $G$ as coset representatives of $H$ in $G$ from
<groupname.cosname>'.diff1' or <groupname.cosname>'.diff2', rather than from
<groupname>.<cosname>'.kbprog'. These require 'autcos' or its
constituents (see Chapter 
"The Automatic Coset System and Subgroup Presentation Package")
to have been run first on the subgroup $H$ of $G$.
\item[|-f |] Save space where possible (at the cost of increased cpu-time)
when running the program 'gpchecksubwa', by calling its options
'-ip s' and '-f' (see below).
\end{description}
The remaining options are standard.

\Section{gpmakesubwa}

'gpmakesubwa [-w] [-diff1/-diff2/-diff1c] [-kbprogcos/-diff1cos/-diff2cos]'\\
'| || || || || || || || || || || || |[-mrl <maxreducelen>] [-nc nochange]\
[-l] [-v] [-silent]'\\
'| || || || || || || || || || || || |[-ms <maxstates>] <groupname>\
[<subname>]'

'gpmakesubwa' constructs a candidate $W_H$ for the subgroup word-acceptor
for the subgroup $H$ of the short-lex automatic group $G$. This is the first
of the two programs run by the Bourne-shell script 'gpsubwa' (see above).
Its input and output files, and several of the options are the same as
described above for 'gpsubwa'. The additional options are as follows.

*Options*\\
\begin{description}
\item[|-w |] This must be used if 'gpmakesubwa' and 'gpchecksubwa' have already
been run on this example, and 'gpchecksubwa' has reported that the
current candidate word-acceptor, $W_H$, constructed by 'gpmakesubwa' is
incorrect. 'gpchecksubwa' will then have output some $G$-reduced words that
lie in $H$ but are not accepted by $W_H$ to the file
<groupname.subname>'.words'.  These words are then read in by 'gpmakesubwa',
and it will ensure that they are accepted by the next candidate constructed.
\item[|-nc| <nochange>]| |\newline
$W_H$ is computed by considering $G$-reduced words
that lie in $H$, by repeatedly multiplying the generators of $H$ and
any additional words output by 'gpchecksubwa' together, and altering
$W_H$ if necessary so that it accepts these words.
When <nochange> such words have been considered without causing any
change in $W_H$, the program halts and outputs $W_H$.
The initial default is 64, but this is increased to a
maximum of 4096 if additional words output by 'gpchecksubwa' are used.
It can be made even higher if required by using this option.
\item[|-ms| <maxstates>]| |\newline
This resets the maximum number of states allowed in the candidate subgroup
word-acceptor from the default (32767) to <maxstates>.
Again, it is unlikely for this to be necessary.
\item[|-mrl| <maxreducelen>]| |\newline
This resets the maximul length of words allowed in the program from the default
(4095) to <maxreducelen>. It is unlikely for this to be necessary.
\end{description}

\Section{gpchecksubwa}

'gpchecksubwa [-ip d/s[<dr>]] [-op d/s] [-l] [-v] [-silent] [-f]\
[-m <maxwords>]'\\
'| || || || || || || || || || || || || |<groupname> [<subname>]'

'gpchecksubwa' reads in a candidate word acceptor $W_H$ produced by
'gpmakesubwa' from the file <groupname.subname>'.wa', and checks it for
correctness. If it finds it is correct, then it exits with exit-status 0.
It it finds it is incorrect, then it exits with status 2, and also
outputs a list of $G$-reduced words that lie in $H$ but are not accepted
by $W_H$ to the file <groupname.subname>'.words'.
(As usual, exit status 1 is used when some kind of fatal error, such as
missing input file, or syntax error in input file occurs.)
The general multiplier for the short-lex automatic structure of $G$ is
required, and is input from <groupname>'.gm'. A number of temporary
multiplier files for $G$, like those produced by 'gpaxioms' (see
Section "gpaxioms"), are computed and output, but are removed before
the program exits (assuming it exits normally).

*Options*\\
These are all standard, except for the following.
\begin{description}
\item[|-f |] This is the same as for 'gpmakefsa' or 'gpaxioms'. See Sections
"gpmakefsa" and "gpaxioms".
\item[|-m| <maxwords>]| |\newline
Abort after finding <maxwords> $G$-reduced words that lie in $H$ but are
not accepted by $W_H$. The default is 2048.
\end{description}

%%File fsa.tex
\Chapter{Programs for Manipulating Finite State Automata}
In this chapter, we describe the utility functions that are provided for
manipulating finite state automata.
All of these except 'fsafilter', 'nfadeterminize' and 'midfadeterminize'
accept only deterministic automata as input.
For explanation of all of the standard options,
see Section "Exit Status of Programs and Meanings of Some Options".

{\em Acknowledgement}\:\ 
The code for 'fsagrowth' was written by Laurent Bartholdi

\Section{fsafilter}

'fsafilter [-ip d/s[<dr>]] [-op d/s] [-silent] [-v] [-csn] [<filename>]'

This merely reads in a finite state automaton and prints it out again.
It is used partly for testing, but also if one wants to change the format
of an automaton from dense to sparse or vice-versa.
If the optional <filename> argument is present, then input is from <filename>
and output to <filename>'.filter'. Otherwise, input is from 'stdin' and
output to 'stdout'.

If the '-csn' option is called, then the numbers of the states in the
transition table of the output are included as comments after the lists
of transitions for each state. This can help to increase the readability
of a file defining an automaton with a large number of states.

\Section{fsamin}

'fsamin [-ip d/s] [-op d/s] [-silent] [-v] [-l] [<filename>]'

A finite state automaton is read in, minimized and then printed out again.
If the optional <filename> argument is present, then input is from <filename>
and output to <filename>'.min'. Otherwise, input is from 'stdin' and
output to 'stdout'.

\Section{fsalabmin}

'fsalabmin [-ip d/s] [-op d/s] [-silent] [-v] [-l] [<filename>]'

This is a minimization program for finite state automata which may have
more than the standard two state categories (accepting and non-accepting),
and the output automataon has a minimal number of states subject to
the condition that any word read by the initial and output automata
leads to states having the same category.

The input automaton must have state set of type  <labeled>, and the labels
of the staes are used to specify the categories. One of these categories
corresponds to label 0, which means no label. Preferably, the accepting states
of the automaton should consist of a union of those states having certain
labels. If this is not the case, then the accepted languages of the input
and output automata will not necessarily be the same.

If the optional <filename> argument is present, then input is from <filename>
and output to <filename>'.labmin'. Otherwise, input is from 'stdin' and
output to 'stdout'.

\Section{fsabfs}

'fsabfs [-ip d/s] [-op d/s] [-silent] [-v] [-l] [<filename>]'

A finite state automaton is read in, and then its states are permuted into
bfs-order (bfs = breadth-first-search), and it is printed out again.
This means that the states are numbered $1,2, \ldots, ns$, and if one
scans the transition-table, in order of increasing states, then the states
occur in increasing order. 'fsamin' and 'fsabfs' used together can be used
to check whether two deterministic automata with the same alphabet
have the same language.
First apply 'fsamin' and then 'fsabfs'.
If they have the same language, then the resulting automata should be identical.
If the optional <filename> argument is present, then input is from <filename>
and output to <filename>'.bfs'. Otherwise, input is from 'stdin' and
output to 'stdout'.

\Section{fsacount}

'fsacount  [-ip d/s] [-silent] [-v] [<filename>]'

A finite state automaton is read in,
the size of the accepted language is counted, and the answer (which may
of course be infinite) is output to 'stdout'.
Input is from <filename> if
the optional argument is present, and otherwise from 'stdin'.

\Section{fsaenumerate}

'fsaenumerate  [-ip d/s] [-bfs/-dfs] <min> <max> [<filename>]'

A finite state automaton is read in from 'stdin' or from the file <filename>
if the optional argument is present.
<min> and <max> should be non-negative integers with <min> $\le$ <max>.
The words in the accepted language having lengths at least <min> and at
most <max> are enumerated, and output as a list of words to 'stdout' or
to the file <filename>'.enumerate'.

If the option '-dfs' is called (depth-first search -- the default),
then the words
in the list will be in lexicographical order, whereas with '[-bfs]'
(breadth-first-search), they will be in order of increasing length, and in
lexicographical order for each individual length (i.e. in shortlex order).
Depth-first-search is marginally quicker.

\Section{fsalequal}

'fsalequal [-ip d/s] [-silent] [-v] <filename1> <filename2>'

Two finite state automata are read in from the files <filename1> and
<filename2>. This program tests whether they have the same language.
The exit code is 0 if they do, and 2 if not.
(To have the same language they must have identical alphabets.)

\Section{fsaand}

'fsaand [-ip d/s[<dr>]] [-op d/s] [-silent] [-v] [-l]\
 <filename1>'\\
'| || || || || || || |<filename2> <outfilename>'

Two finite state automata, which must have the same alphabet,
are read in from the files <filename1> and <filename2>. An automaton that
accepts a word $w$ in the alphabet if and only if both of the input automata
accept $w$ is computed, minimized, and output to the file <outfilename>.

\Section{fsaor}

'fsaor [-ip d/s[<dr>]] [-op d/s] [-silent] [-v] [-l]\
 <filename1>'\\
'| || || || || || |<filename2> <outfilename>'

Two finite state automata, which must have the same alphabet, are read in from
the files <filename1> and <filename2>. An automaton that accepts a word $w$ in
the alphabet if and only if at least one of the input automata
accept $w$ is computed, minimized, and output to the file <outfilename>.

\Section{fsaandnot}

'fsaandnot [-ip d/s[<dr>]] [-op d/s] [-silent] [-v] [-l]\
 <filename1>'\\
'| || || || || || |<filename2> <outfilename>'

Two finite state automata, which must have the same alphabet, are read in from
the files <filename1> and <filename2>. An automaton that accepts a word $w$ in
the alphabet if and only if the first input automaton
accepts $w$ and the second input automaton does not accept $w$
is computed, minimized, and output to the file <outfilename>.

\Section{fsanot}

'fsanot [-ip d/s[<dr>]] [-op d/s] [-silent] [-v] [-l] <filename>'

A finite state automaton is read in from the file <filename>, and an automaton
with the same alphabet, which accepts a word $w$ if and only if the input
automaton does not accept $w$, is calculated and output to <filename>'.not'.

\Section{fsaconcat}

'fsaconcat [-ip d/s[<dr>]] [-op d/s] [-silent] [-v] <filename1>'\\
'| || || || || || |<filename2> <outfilename>'

Two finite state automata, which must have the same alphabet, are read in from
the files <filename1> and <filename2>. An automaton that accepts a word $w$ in
the alphabet if and only if $w = w_1w_2$, where $w_1$ and $w_2$ are in
the respective languages of the input automata,
is computed, minimized, and output to the file <outfilename>.

\Section{fsastar}

'fsastar [-ip d/s[<dr>]] [-op d/s] [-silent] [-v] <filename>'

A finite state automaton is read in from the file <filename>, and an automaton
with the same alphabet, which accepts a word $w$ if and only if 
$w \in L^\*$, where $L$ is the language of the input automaton,
is calculated, minimized, and output to <filename>'.star'.

\Section{fsareverse}

'fsareverse [-s] [-midfa] [-ip d/s[<dr>]] [-op d/s] [-silent] [-v] [-l]'\\
'| || || || || || |<filename>'

A finite state automaton is read in from the file <filename>, and an automaton
with the same alphabet, which accepts a word $w$ if and only if the input
automaton accepts the reversed word $w^R$, is calculated and output to
<filename>'.reverse'.

If '-s' is used, then the states of the output automaton (which will
not be minimized) will be labeled as lists of integers, which specify
the corresponding subsets of the states of the input automaton.

If '-midfa' is used, then the output automaton will be a midfa,
and the initial states  will correspond to the accepting states of the
input automaton.

\Section{fsaexists}

'fsaexists [-ip d/s[<dr>]] [-op d/s] [-silent] [-v] [-l] <filename>'

A two-variable finite state automaton is read in from the file <filename>,
and a one-variable automaton, which accepts a word $v$ if and only if the input
automaton accepts $(v,w)$ for some word $w$,
is calculated and output to <filename>'.exists'.

\Section{fsaswapcoords}

'fsaswapcoords [-op d/s] [-silent] [-v] [-l] [<filename1> <filename2>]'

A two-variable finite state automaton is read in from <stdin> or from
the file <filename1> if specified. A new automaton that accepts a pair of
words $(v,w)$ if and only if the original automataon accepts $(w,v)$ is
computed, and output to 'stdout' or to the file <filename2> if specified.
(This could, for example, be used before 'fsaexists', if it was desired to
quantify over the second coordinate of a two-variable automataon, rather
than the first.)

\Section{fsaprune}

'fsaprune [-ip d/s] [-op d/s] [-silent] [-v] [-a] [<filename>]'

A finite state automaton is read in, and a modified automaton computed
and then printed out. The output automaton accepts a word $w$ if and
only if the input automaton accepts $w$ and infinitely many words having
$w$ as a prefix.

If |-a| is used, then the computation starts by making all states of the
automaton accepting (and the program runs quicker with this option set).
Equivalently, with |-a|, the output automaton accepts a word $w$ if and
only if the input automaton accepts infinitely many words having
$w$ as a prefix.

This program is useful, for example, for finding geodesics to the boundary in
word-hyperbolic groups.

If the optional <filename> argument is present, then input is from <filename>
and output to <filename>'.prune'. Otherwise, input is from 'stdin' and
output to 'stdout'.

\Section{fsagrowth}

'fsagrowth [-op d/s] [-silent] [-v] [-primes x,y,...] <filename>'

'fsagrowth' computes the rational growth function of a finite state automaton.
The input is from <filename> (or from 'stdin' if the optional argument is not
present), which should contain a finite state automaton.
The output is to <filename>'.growth' or to 'stdout', and contains the quotient
of two integral polynomials in the variable $X$. The coefficient  of $X^n$ in
the Taylor expansion of this polynomial is equal to the number of accepted
words of the automaton length $n$.

In fact, to avoid the danger of integer overflow, the calculations are not done
over the integers, but modulo a sequence of primes. If, on lifting to the
integers, different primes give different results, then the coefficients of
the polynomials are too large, and the results are likely to be wrong.
A warning message is printed when this happens, and the exit code is 2.
By default, three primes slightly less than $2^{15}$ are used, but the
user can input the primes to be used by using the '-primes x,y,...' option.
The list of primes here should be comma-separate, with no space.

The code for this program was written by Laurent Bartholdi.

\Section{nfadeterminize}
'nfadeterminize [-s] [-op d/s] [-silent] [-v] [-l] [<filename>]'

'nfadeterminize' read a (possibly) non-deterministic automaton from
'stdin', or from <filename> if the optional argument is present.
It computes and minimizes an equivalent deterministic
automaton with the same language and outputs the result to 'stdout'
or to <filename>'.determinize'. The input automaton, which will
normally be stored in sparse format, may contain $\varepsilon$-transitions,
which should be specified by giving them the generator number 0.

If the '-s' option is called, then the states of the output automaton
(which will not be minimized) will be labeled as lists of integers, which
specify the corresponding subsets of the states of the input automaton.
The other options are standard.

\Section{midfadeterminize}

'midfadeterminize [-ip d/s[<dr>]] [-op d/s] [-silent] [-v] [-l] [<filename>]'

This is in fact a special case of 'nfadeterminize'.
A midfa (multiple initial state deterministic finite state automaton) is read
from 'stdin' or <filename>, and a minimized deterministic automaton that
accepts the same language is output to 'stdout' or to
<filename>'.midfadeterminize'.

%%File gap.tex
\Chapter{The Interface with GAP}

The information in this chapter refers to Version 3 of {\GAP}.
No attempt has been made yet to link t{\KBMAG} to Version 4.
This will hopefully happen eventually.

There are two ways in which this interface can be used.
In the first, files are read into {\GAP} that have been created externally.
A {\GAP} conversion function 'ReadRWS' has been provided for this purpose.
This function assumes that the name of the rewriting system (i.e. the left
hand side of the declaration contained in the file) is |_RWS|, so this
name should always be used for files that are to be accessed by {\GAP}.

The second only works currently for groups (since monoids do not yet
exists as a {\GAP} type). Documentation for this method can also be found as
a chapter in the {\GAP} manual in the 'gapdoc' directory of {\KBMAG}.
The applications of {\KBMAG} involving subgroups of groups; that is those
documented in Chapters
"The Knuth--Bendix Program for Cosets of Subgroups",
"The Automatic Coset System and Subgroup Presentation Package" and
"Creating Subgroup Word Acceptors"
of this manual, have not yet been incorporated within the interface.
A finitely presented group, $G$ say, is first
defined within {\GAP}, and then the {\GAP} function 'FpGroupToRWS' is called
on $G$, and a corresponding rewriting system is returned.
It can be called with an optional boolean second-variable, for example

|R := FpGroupToRWS(G,true);|

in which case inverses of generators are
printed using a case-change convention. Otherwise, inverses are printed in
the usual way by appending the suffix |^-1| to the generator name.

Since the internal storage of rewriting-systems in particular is
different from the structure defined in the external file (for example,
words are stored internally as lists of integers), the user should not
attempt to create rewriting systems directly internally to {\GAP},
but should either read them in from external files, or create them using the
function 'FpGroupToRWS'.

An implementation of finitely presented monoids within {\GAP} is currently
being written, and this too will eventually have an interface with
rewriting-systems.
 
The library files in the directory called 'gap' contain elementary functions
for manipulating finite state automata and rewriting-systems.
Anyone who wishes to use these seriously, should look in the files 'fsa.g' and
'rws.g'. A finite state automaton must either be typed in
interactively, or (better) put into a file first, and read in. Of course,
the automatic group programs calculate finite state automata, and the
functions are probably mainly useful for playing with these. It important to
know that, before using any of the other {\GAP} functions on an automaton <fsa>,
the function 'InitializeFSA(<fsa>)' must be called. Some of the functions,
like those that count or enumerate the language of a finite state automaton,
perform the same operations as the corresponding standalone 'C'-programs,
but they will usually be much slower than the standalones. Nevertheless,
it is often convenient to be able to do such things without having to go
through the process of writing to file, running external program, and then
reading the answer back in.

After installing the package, but before starting to use the {\GAP} functions,
go into the {\GAP} directory and type 'makeinit'. This will create the
required 'init.g' file for the {\GAP} library.
This file should be read in from {\GAP} before using the functions.

The use of the library is best illustrated by providing a report of an
actual session, with comments interspersed.
The first example illustrates the use of 'kbprog' from {\GAP}.
First create the following external file called 'S3' (which, we shall assume
is in the directory |../kb_data|).

|
_RWS := rec(
           isRWS := true,
  generatorOrder := [a,b],
        inverses := [a,b],
        ordering := "shortlex",
       equations := [
         [b*a*b,a*b*a]
       ]
);
|

Now start {\GAP} and proceed as follows.

|
gap> Read("init.g");
gap> S3 := ReadRWS("../kb_data/S3");
#Reads in the file, and converts it to an internal rewriting-system.
#However, it is still displayed as in the external file!
rec(
           isRWS := true,
  generatorOrder := [a,b],
        inverses := [a,b],
        ordering := "shortlex",
       equations := [
         [b*a*b,a*b*a]
       ]
)

gap> #Now we run the Knuth-Bendix program using the function 'KB', which
gap> #calls the external program "kbprog".

gap> KB(S3);
#System is confluent.
#Halting with 3 equations.
true
gap> #S3 now contains the confluent set of equations.
gap> S3;
rec (
           isRWS := true,
     isConfluent := true,
        ordering := "shortlex",
  generatorOrder := [a,b],
        inverses := [a,b],
       equations := [
         [a^2,IdWord],
         [b^2,IdWord],
         [b*a*b,a*b*a]
       ]
)

gap> #Now we can do some word-reductions.
gap> ReduceWordRWS(S3,(a*b)^3);
IdWord
gap> ReduceWordRWS(S3,a*b*a);
a*b*a
gap> ReduceWordRWS(S3,b*a*b*a);
a*b
gap> SizeRWS(S3);
6
gap> EnumerateRWS(S3,0,5);
[ IdWord, a, a*b, a*b*a, b, b*a ]
|

The next example illustrates the use of the automatic group package from {\GAP}.
This example is not quite so trivial as the preceding one. The group is the
fundamental group of the complement of the Borromean rings, which is a
three-dimensional hyperbolic manifold. The word-acceptor was used to enumerate
the words of length up to 4. This was used to speed up the calculations
required for drawing views of a tessellation of hyperbolic space by regular
dodecahedra, which can be seen in the video ``Not Knot\'\'.

First make the following file, called |BR| (in the directory |../ag_data|).
|
_RWS := rec(
 isRWS := true,
 ordering := "shortlex",
 generatorOrder := [a,A,b,B,c,C,d,D,e,E,f,F],
 inverses := [A,a,B,b,C,c,D,d,E,e,F,f],
 equations := [
  [a*a*a*a,IdWord], [b*b*b*b,IdWord], [c*c*c*c,IdWord],
  [d*d*d*d,IdWord], [e*e*e*e,IdWord], [f*f*f*f,IdWord],
  [a*b*A*e,IdWord], [b*c*B*f,IdWord], [c*d*C*a,IdWord],
  [d*e*D*b,IdWord], [e*f*E*c,IdWord], [f*a*F*d,IdWord]
 ]
);
|

Now start {\GAP} and proceed as follows.

|
gap> Read("init.g");
gap> BR := ReadRWS("../ag_data/BR");
rec(
           isRWS := true,
  generatorOrder := [a,A,b,B,c,C,d,D,e,E,f,F],
        inverses := [A,a,B,b,C,c,D,d,E,e,F,f],
        ordering := "shortlex",
       equations := [
         [a^4,IdWord],
         [b^4,IdWord],
         [c^4,IdWord],
         [d^4,IdWord],
         [e^4,IdWord],
         [f^4,IdWord],
         [a*b*A*e,IdWord],
         [b*c*B*f,IdWord],
         [c*d*C*a,IdWord],
         [d*e*D*b,IdWord],
         [e*f*E*c,IdWord],
         [f*a*F*d,IdWord]
       ]
)

gap> #Now we run the automatic group program using the function 'AutGroup',
gap> #which calls the external program "autgroup".

gap> AutGroup(BR);

.....   (output omitted)   .....

gap> SizeRWS(BR);
"infinity"

gap> #First do a few reductions.
gap> ReduceWordRWS(BR,a*B*c*D*e*F);
a*B*a*c^2*e
gap> ReduceWordRWS(BR,(a*c*e)^5);
a*c*a*B*c*a*B*c*a*B*c*a*B*c*e

gap> #Now we enumerate words in the group up to length 4.
gap> #``SortEnumerate\'\'\ puts them in order of increasing length.
gap> SortEnumerateRWS(BR,0,4);
[ IdWord, a, A, b, B, c, C, d, D, e, E, f, F, a^2, a*b, a*B, a*c, a*C, a*d, 
  a*D, a*e, a*E, a*f, a*F, A*b, A*B, A*c, A*C, A*d, A*D, A*e, A*E, A*f, A*F, 
  b*a, b^2, b*c, b*C, b*d, b*D, b*e, b*E, b*f, b*F, B*a, B*c, B*C, B*d, B*D, 
  B*e, B*E, B*f, B*F, c*a, c*A, c*b, c^2, c*e, c*E, c*f, c*F, C*a, C*A, C*b, 
  C*d, C*D, C*e, C*E, C*f, C*F, d*a, d*A, d*b, d*B, d*c, d^2, d*f, d*F, D*a, 
  D*A, D*b, D*B, D*c, D*e, D*E, D*f, D*F, e*A, e*b, e*B, e*c, e*C, e*d, e^2, 
  E*A, E*b, E*B, E*c, E*C, E*d, E*f, E*F, f*B, f*c, f*C, f*d, f*D, f*e, f^2, 
  F*a, F*A, F*B, F*c, F*C, F*e, a^2*b, a^2*B, a^2*c, a^2*C, a^2*d, a^2*D, 

  .....  (much output omitted)   .....

  F*e*b*f, F*e*b*F, F*e*B*a, F*e*B*c, F*e*B*C, F*e*B*d, F*e*B*D, F*e*B*e, 
  F*e*B*E, F*e*B*f, F*e*B*F, F*e*c*a, F*e*c*A, F*e*c*b, F*e*c^2, F*e*c*E, 
  F*e*c*f, F*e*c*F, F*e*C*A, F*e*C*b, F*e*C*d, F*e*C*D, F*e*C*E, F*e*C*f, 
  F*e*C*F, F*e*d*a, F*e*d*A, F*e*d*b, F*e*d*B, F*e*d*c, F*e*d*F ]

gap> #Finally, let\'s see how many words there are of different lengths.
gap> for ct in [0..6] do
> Print(SizeEnumerateRWS(BR,ct,ct),"\n");
> od;
1
12
102
812
6402
50412
396902
gap> quit;
|

{\bf UPDATE}\:\ It is now possible to install
{\KBMAG} as a share-libraryin Version 3 of {\GAP}.
Instructions for this are in the 'README' file in
the main {\KBMAG} directory.

{\bf NEW}\:\ The file 'wordorder.g' written by Sarah Rees contains some
experimental routines for finding the automatic structure of groups using
orderings other than short-lex, such as weighted lex, and inverse-pair
wreath product. There is currently at least one bug in this, in that
they will not complete successfully if a generator reduces (under
this ordering) to a word of length greater than one. The interested
user is advised to seek help by e-mail!



%%%%%%%%%%%%%%%%%%%%%%%%%%%%%%%%%%%%%%%%%%%%%%%%%%%%%%%%%%%%%%%%%%%%%%%%%%%%%
%%
%%  and the bibliography
%%
\begin{sloppypar}
\bibliographystyle{alpha}
\newcommand{\ignore}[1]{}
\catcode`\'=12 \catcode`\<=12 \catcode`\*=12
\catcode`\|=12 \catcode`\:=12 \catcode`\"=12
\bibliography{manual}
\end{sloppypar}

\end{document}
