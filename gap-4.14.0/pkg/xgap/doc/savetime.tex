\Chapter{How to read this manual}

This chapter tells you exactly, which part of this manual you have to read
if you want to learn certain things about \XGAP.

If you do not know anything about {\XGAP} you might want to have a look at
chapter "xgap:What is XGAP?". You can learn there, what {\XGAP} is all
about, what you can do with it as a user (doing no programming) and for
what you can use {\XGAP} within your own programs. You find also references
to other sections of this manual to get into the details.

If you want to know how to install \XGAP, then you should look into 
chapter "xgap:Installing XGAP".

If you know {\XGAP} from its \GAP3 version, you will consider chapter
"xgap:Differences to XGAP 3" useful. There you can quickly ``update'' your
knowledge to the new \GAP4 version and also find some technical details
about the implementation, which is nearly completely new.

Perhaps you know already, that you can display subgroup lattices
interactively with {\XGAP} and you want to start with this right now. In
this case, jump to chapters "xgap:Subgroup Lattices - Examples" and
"xgap:Subgroup Lattices - systematic description"
immediately. However, you should make sure first that you have a working
{\XGAP} installed! 

The {\XGAP} library is described in these and the remaining chapters of this
manual. The information is divided into the following parts:
\beginitems
*subgroup lattices - examples* &
  
*subgroup lattices - systematic description* &

*graphic sheets* &
  Creating and managing graphic sheets and graphic objects. This is the
  lowest level.

*user communication* &
  Handling of menus, dialogs, text selectors and popups.

*graphic posets* &
  Display of posets in graphic sheets.

*graphic graphs* &
  Display of mathematical graphs in graphic sheets. A lot of ``routine
  work'' for handling these things is already done and can be used in
  your applications!
\enditems
There is a chapter for each of these parts.



