
%%%%%%%%%%%%%%%%%%%%%%%%%%%%%%%%%%%%%%%%%%%%%%%%%%%%%%%%%%%%%%%%%%%%%%%%%%%%%
\Chapter{The organization of the data}

We include some brief comments on the organization of the data. As a 
preliminary step we briefly recall the $p$-group generation algorithm.

%%%%%%%%%%%%%%%%%%%%%%%%%%%%%%%%%%%%%%%%%%%%%%%%%%%%%%%%%%%%%%%%%%%%%%%%%%%%%
\Section{The $p$-group generation algorithm}

The $p$-group generation algorithm was developed and implemented by Eamonn
O'Brien, and we refer the reader to \cite{OBrien90} for a detailed
description of the algorithm. 

For a brief overview, let $P$ be a $p$-group. The algorithm uses the lower 
$p$-central series, defined recursively by $\lambda_{1}(P)=P$ and 
$\lambda_{i+1}(P)=[\lambda_{i}(P),P]\lambda_{i}(P)^{p}$ for $i\geq 1$. 
The $p$-class of $P$ is the length of this series. Each $p$-group $P$, 
apart from the elementary abelian ones, is an immediate descendant of 
the quotient $P/R$ where $R$ is the last non-trivial term of the lower 
$p$-central series of $P$. 

Thus all the groups with order $3^{8}$, except the elementary abelian one, 
are immediate descendants of groups with order $3^{k}$ for some $k$ smaller
than $8$. All of 
the immediate descendants of a $p$-group $Q$ are quotients of a certain 
extension of $Q$; the isomorphism problem for these descendants is equivalent
to the problem of determining orbits of certain subgroups of this extension 
under an action of the automorphism group of $Q$. Not all $p$-groups have 
immediate descendants, those that do are called capable, and those which 
do not are called terminal. 

O'Brien and Vaughan-Lee's classification of the groups of order $p^{7}$ 
in \cite{OVL05} is based on a classification of the nilpotent Lie rings of 
order $p^{7}$, and the groups of order $p^{7}$ are obtained from the Lie 
rings using the Baker-Campbell-Hausdorff formula. O'Brien and Vaughan-Lee 
classified the nilpotent Lie rings of order $p^{7}$ using the nilpotent 
Lie ring generation algorithm, which is a direct analogue of the $p$-group 
generation algorithm. 

Thus the databases of nilpotent Lie rings of order 
$p^{7}$ and of the groups of order $3^{8}$ are organized according to these 
algorithms: the immediate descendants of order $p^{7}$ of each nilpotent 
Lie ring of order less than $p^{7}$ are grouped together in the database of 
nilpotent Lie rings, and the immediate descendants of order $3^{8}$ of each 
group of order less than $3^{8}$ are grouped together in the database of 
groups of order $3^{8}$.

%%%%%%%%%%%%%%%%%%%%%%%%%%%%%%%%%%%%%%%%%%%%%%%%%%%%%%%%%%%%%%%%%%%%%%%%%%%%%
\Section{The groups of order 6561}

The database of groups of order $3^{8}$ is organized according to rank and 
$p$-class. Here rank is the rank of the Frattini quotient, i.e. the size of
a minimal generating set, and $p$-class is as defined in the previous
section. The following table gives the number of groups of order $3^{8}$ of
each rank and $p$-class, with the $(i,j)$ entry corresponding to rank $i$
and $p$-class $j$.

\beginexample
  | 1   2        3        4       5      6     7   8
--|-------------------------------------------------
1 | 0   0        0        0       0      0     0   1 
2 | 0   0        58       486     1343   330   9   0 
3 | 0   4        216747   40521   2163   24    0   0 
4 | 0   23361    494666   22343   51     0     0   0
5 | 0   578478   14796    80      0      0     0   0 
6 | 0   566      39       0       0      0     0   0 
7 | 0   10       0        0       0      0     0   0 
8 | 1   0        0        0       0      0     0   0
\endexample

In the list of all groups of order $3^8$ the first group is the cyclic
group, then the 2-generator groups follow in order of increasing $p$-class
and so on. The above table can thus be used to find the range of numbers
of groups with a given rank and $p$-class.

As mentioned above, the database is organized according to the $p$-group
generation algorithm. For example, the 9 groups of rank 2, $p$-class 7, and
order $3^{8}$ are numbered from 2219--2227. The groups numbered 2219 and
2220 are descendants of SmallGroup($3^{7},384$), and the groups numbered
2221--2227 are descendants of SmallGroup($3^{7},386$). Similarly, the 24
groups of rank 3, $p$-class 6, and order $3^{8}$ are numbered from
261663--261686. The first four of these groups are descendants of 
SmallGroup($3^{7},5841$), the next 17 are descendants of 
SmallGroup($3^{7},5844$), and the last 4 are descendants of 
SmallGroup($3^{7},5849$).

%%%%%%%%%%%%%%%%%%%%%%%%%%%%%%%%%%%%%%%%%%%%%%%%%%%%%%%%%%%%%%%%%%%%%%%%%%%%%
\Section{The groups with order the seventh power of a prime}

The groups of order $p^{7}$ for prime $p>11$ are obtained from the LiePRing
database of nilpotent Lie rings of order $p^{7}$ using Willem de Graaf's
implementation of the Baker-Campbell-Hausdorff formula. The LiePRing
database is organized according to the output from the nilpotent Lie ring
generation algorithm. For any given $p$ the first Lie ring in the database
is the cyclic Lie ring of order $p^{7}$. Next come the two generator Lie
rings, then the three generator Lie rings, and so on, ending with the six
generator Lie rings, and then finally the elementary abelian Lie ring of
rank 7. The first four of the two generator nilpotent Lie rings of order 
$p^{7}$ are immediate descendants of the Lie ring
$$
\langle a,b\,|\,pb,\,class 3\rangle 
$$
of order $p^{4}$. The next $p^{2}+8p+25$ are immediate descendants of the
Lie ring
$$
\langle a,b\,|\,baa,bab,pba,\,class 3\rangle 
$$
of order $p^{5}$, and the next $p+6+(p^{2}+3p+10) gcd(p-1,3)$ are immediate
descendants of
$$
\langle a,b\,|\,babb,pa,pb,\,class 4\rangle . 
$$
The nine rank 6 Lie rings in the database are the rank 6, $p$-class 2 
Lie rings. These are the immediate descendants of the elementary abelian 
Lie ring of rank 6.

There is a complete list of presentations for the nilpotent Lie rings of
order $p^{k}$ for $k\leq 7$ valid for all $p>3$ in the document p567.pdf
supplied with the documentation for the LiePRing package. The presentations
are grouped as described above, with each group of presentations giving the
immediate descendants of a Lie ring of smaller order.

In a few cases the descendants of a parametrized family of Lie rings are
grouped together. For example there is a family of $p(p-1)$ distinct Lie
rings with presentations of the form
$$
\langle a,b,c\,|\,ca-baa,\,cb,\,pa-\lambda baa-\mu bab,\,pb+\nu baa+\xi
bab,\,pc,\,class 3\rangle 
$$
with $\lambda ,\mu ,\nu ,\xi \neq 0$. Most of these algebras are terminal,
but $\frac{5}{2}p-\frac{9}{2}+\frac{1}{2}\gcd (p-1,4)$ of them are capable,
and they have a total of
$\frac{1}{2}p^{3}+2p^{2}-5p+\frac{1}{2}+\frac{p}{2}\gcd (p-1,4)$
descendants of order $p^{7}$ and $p$-class 4. These
descendants have presentations

$$
\langle a,b,c\,|\,ca-baa,\,cb,\,pa-baa-\mu bab-ybaaa,\,pb+\nu
baa+bab-zbaaa,\,pc-tbaaa,\,class 4\rangle , 
$$

$$
\langle a,b,c\,|\,ca-baa,\,cb,\,pa-baa-\mu bab-ybaaa,\,pb+\nu baa+\mu \nu
bab-zbaaa,\,pc-tbaaa,\,class 4\rangle 
$$

for various choices of the parameters $\mu ,\nu ,y,z,t$. For any given value
of $p$ the $\frac{1}{2}p^{3}+2p^{2}-5p+\frac{1}{2}+\frac{p}{2}\gcd (p-1,4)$
distinct Lie algebras with presentations of this form are grouped together,
with consecutive numbering.

There is no easy way to determine the numbering of (say) the three generator
Lie rings of $p$-class 4 since the numbers depend on $p$ in a very
complicated way, and generally there is no easy efficient way of searching
the database for a group with particular properties. In view of the numbers
of groups of order $p^{7}$ and the time needed to generate a complete list,
this means that the database will be of limited use for most people. A user
who wants to access a particular batch of descendants as described in
p567.pdf is advised to use the LiePRing package directly, as this package
also has an option to obtain the corresponding groups via Willem de Graaf's
implementation of the Baker-Campbell-Hausdorff formula. On the other hand,
there are only 

$$
2p^{3}+13p^{2}+64p+145+(p^{2}+10p+56) gcd(p-1,3)+(4p+28) gcd(p-1,4)
$$
$$
 +(2p+12) gcd(p-1,5)+ gcd(p-1,7)+4 gcd(p-1,8)+ gcd(p-1,9)
$$
two generator groups of order $p^{7}$ for $p>5$ and a complete list of them
can be generated quite quickly for moderate values of $p$. The table below
gives the number of $d$ generator groups of order $p^{7}$ valid for all 
$p>5$, and the user can use the table to compute the range of numbers needed 
to access the $d$ generator groups of order $p^{7}$ for any given $p$.

\beginitems
<rank 1> & $1$ 

<rank 2> 
    & $2p^{3}+13p^{2}+64p+145+(p^{2}+10p+56)gcd (p-1,3) +(4p+28)gcd (p-1,4)$
    & $+(2p+12)gcd (p-1,5)+gcd (p-1,7)+4gcd (p-1,8)+gcd (p-1,9)$

<rank 3> 
      & $2p^{5}+9p^{4}+29p^{3}+99p^{2}+380p+1100+(3p^{2}+28p+189)gcd (p-1,3)$
      & $+(p^{2}+13p+84)gcd (p-1,4)+(p+17)gcd (p-1,5)+gcd (p-1,8)+3gcd (p-1,7)$

<rank 4> 
      & $p^{5}+3p^{4}+13p^{3}+57p^{2}+248p+1044+(6p+46)gcd (p-1,3)$ 
      & $+(2p+23)\gcd (p-1,4)+2\gcd (p-1,5)$ 

<rank 5> & $p^{2}+15p+155$ 

<rank 6> & $9$ 

<rank 7> & $1$
\enditems

The total number of groups of order $p^{7}$ for all $p > 5$ is given by

$$
3p^{5}+12p^{4}+44p^{3}+170p^{2}+707p+2455
$$
$$
+(4p^{2}+44p+291)gcd (p-1,3)+(p^{2}+19p+135)gcd (p-1,4) 
$$
$$
+(3p+31)gcd (p-1,5)+4gcd (p-1,7)+5gcd (p-1,8)+gcd (p-1,9).
$$

